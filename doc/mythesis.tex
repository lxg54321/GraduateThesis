% Copyright (c) 2008-2009 solvethis
% Copyright (c) 2010-2011 Casper Ti. Vector
% Public domain.

\documentclass[UTF8, oneside]{pkuthss}

% 使得打字机粗体可以被使用。
\usepackage{lmodern}
% 产生 originauth.tex 里的 \square。
\usepackage{amssymb}
% 提供 Verbatim 环境和 \VerbatimInput 命令。
\usepackage{fancyvrb}

% 解决无色彩时无法生成图片的问题
\usepackage{bmpsize}
% 控制图片位置
\usepackage{float}
% 插入代码
\usepackage{listings}
\lstset{
    tabsize = 4,
    frame = shadowbox, % lines OR shadowbox OR trBL
    numbers = left,
    numberstyle = \tiny,
    %basicstyle = \scriptsize,
    basicstyle = \footnotesize,
    breaklines = true,
    aboveskip = 1em
}

% 参考文献格式。
\bibliographystyle{ref/chinesebst-mod}
% 设定文档的基本信息。
\pkuthssinfo{
	cthesisname={硕士研究生学位论文},ethesisname={Undergraduate Thesis},
	ctitle={一种支持物理层WiFi安全\\研究的验证平台},
	etitle={A Verification platform for Physical Layer WiFi Security Research},
	cauthor={李晓光},
	eauthor={Xiaoguang Li},
	studentid={1401214261},
	date={二〇一七年五月},
	school={信息科学技术学院},
	cmajor={计算机系统结构},emajor={Computer Architecture},
	direction={无线局域网体系结构},
	cmentor={王韬},ementor={Prof.\ Tao Wang},
	ckeywords={WiFi安全,物理层,无线验证平台},
	ekeywords={WiFi security, physical layer, verification platform},
}

% 清理tex local的命令:mktexlsr

\begin{document}
	% 以下为正文之前的部分。
	\frontmatter

	% 自动生成标题页。
	\maketitle

	% 版权声明。
	\include{chap/copyright}
	% 中英文摘要。
	% Copyright (c) 2008-2009 solvethis
% Copyright (c) 2010-2011 Casper Ti. Vector
% Public domain.

\begin{cabstract}

	现代社会中,WiFi作为一种快速、便捷、不考虑流量的无线上网方式,成为人们基本的生活需求,WiFi接入点遍布家庭、企业、公共场所等区域。
	WiFi的普及也为人们的网络生活带来了安全隐患,一些不法分子通过WiFi向用户发起攻击。
	现有的WiFi加密协议包括WEP、WPA和WPA2等存在安全漏洞,存在被破解的可能,
	有的攻击者破解个人WiFi的密码,目的是占用网络资源,降低合法用户的服务质量,俗称“蹭网”。
	另有一些攻击者利用带无线网卡的电脑和网络分析软件搭建钓鱼WiFi,监听网络报文,
	获取用户手机号、网站账号、密码等敏感信息,甚至可以窃取钱财,给用户带来巨大损失。

	为了检测和阻止层出不穷的攻击,针对WiFi安全的研究广泛开展。
	近年来,结合物理层的安全机制成为WiFi安全研究的热点,物理层含有丰富的无线信道、设备位置、信号质量等信息,
	研究者探索如何利用这些信息加强WiFi的安全性。

	WiFi安全的研究需要有开发和验证平台,用以实现和评估研究成果。
	物理层的安全研究对验证平台提出很高的要求,由软件实现的WiFi物理层无法满足协议的要求,而硬件实现往往可编程性较差。
	从近些年的研究看,验证平台的质量常常成为制约研究成果有效性的瓶颈,
	一些验证平台不支持物理层的编程,一些验证平台延迟过高无法与商用WiFi设备实时通信。
	“工欲善其事,必先利其器”,支持物理层编程的高性能WiFi验证平台成为WiFi安全研究必不可少的工具。

	本文对近年来物理层WiFi安全的研究进行了深入分析,总结了这些研究对验证平台的需求,
	设计了一种支持物理层WiFi安全研究的验证平台,并利用商用设备进行了实现,
	通过实际的使用样例论证了可以满足绝大多数物理层WiFi安全研究的需求,
	并基于此平台搭建了伪装WiFi,提出了一种用于识别伪装WiFi的新型WiFi安全认证技术。

\end{cabstract}

\begin{eabstract}

	In modern society, WiFi as a fast, convenient, do not consider the flow of wireless Internet access, as people's basic needs of life, WiFi access point throughout the family, business, public places and other regions.
	WiFi popularity for people's network life has brought security risks, some lawless elements through WiFi to attack the user.
	Existing WiFi encryption protocols include WEP, WPA and WPA2 and other security vulnerabilities, there is the possibility of being cracked.
	Some attackers crack personal WiFi password, the purpose is to take up network resources, reduce the quality of legitimate users of the service, commonly known as "rub network."
	Some other attackers use the wireless network card computer and network analysis software to build fishing WiFi, monitor network packets, access to mobile phone number, website account, password and other sensitive information, and even steal money, to the user a huge loss.

	In order to detect and prevent the endless attacks, for WiFi security research carried out extensively.
	In recent years, the security mechanism combined with the physical layer has become the hotspot of WiFi security research. The physical layer contains rich information such as wireless channel, device location and signal quality. Researchers explore how to use this information to enhance WiFi security.

	WiFi security research requires a development and verification platform to implement and evaluate research results.
	The physical layer of security research on the verification platform made a high demand, by the software to achieve the WiFi physical layer can not meet the requirements of the agreement, and hardware implementation is often poor programmability.
	In recent years, the quality of the verification platform has often become a bottleneck in restricting the effectiveness of research results. Some verification platforms do not support the programming of the physical layer. Some of the verification platforms are too late to communicate with commercial WiFi devices in real time.
	"Good practice, must first of its profits," to support the physical layer programming high-performance WiFi authentication platform for WiFi security research is an indispensable tool.

	In this paper, the research on WiFi security of physical layer in recent years is analyzed, and the requirements of verification platform are summarized. A verification platform supporting the study of WiFi layer security is designed and implemented by commercial device. This paper demonstrates the need to meet the needs of most physical layer WiFi security research, and builds a camouflage WiFi based on this platform, and proposes a new WiFi authentication technology for recognizing camouflage WiFi.

\end{eabstract}

	% 自动生成目录。
	\tableofcontents

	% 以下为正文。
	\mainmatter

	% 各章节。
	% Copyright (c) 2008-2009 solvethis
% Copyright (c) 2010-2011 Casper Ti. Vector
% Public domain.

\chapter{引言}
	\section{课题背景与研究意义}

		现代社会中,WiFi作为一种快速、便捷、不考虑流量的无线上网方式,成为人们基本的生活需求,WiFi接入点遍布家庭、企业、公共场所等区域。
		WiFi的普及为人们的网络生活带来了安全隐患,一些不法分子通过WiFi向用户发起攻击。
		相对于有线网络,无线网络的通信内容广播到空气中,安全性较低。
		现有的WiFi加密协议包括WEP、WPA和WPA2等存在安全漏洞,存在被破解的可能\cite{wisec09wep}。
		有的攻击者破解个人WiFi的密码,目的是占用网络资源,降低合法用户的服务质量,俗称“蹭网”。
		另有一些攻击者利用带无线网卡的电脑和网络分析软件搭建钓鱼WiFi,监听网络报文,
		获取用户手机号、网站账号、密码等敏感信息,甚至可以窃取钱财,给用户带来巨大损失。
		在2015年的央视3·15晚会上,网络安全工程师伪装了演播室的免费WiFi,钓鱼得到的现场观众的信息,
		观众自拍的照片和邮箱密码竟出现在了演播室大屏幕上\cite{cctv315}。
		普通用户有时为了占用更多的无线网络带宽,也会有意或无意地向同网其它用户发起攻击,损害其它用户的网络质量\cite{tifs12reciprocity}。

		为了检测和阻止层出不穷的攻击,针对WiFi安全的研究广泛开展。
		近年来,结合物理层的安全机制成为WiFi安全研究的热点,物理层含有丰富的无线信道、设备位置、信号质量等信息,
		研究者探索如何利用这些信息加强WiFi的安全性,
		例如基于RSS(Received Signal Strength,接收信号强度)和CSI(Channel State Informatica,信道状态信息)的密钥生成策略\cite{access16key},
		基于RSS的物理层认证策略\cite{ieeewc10noncryp},
		基于CIR(Channel Impulse Response,信道冲激响应)的物理层认证技术\cite{milcom11cir, icc13cir},
		基于信道频率响应特性的物理层指纹和窃听检测技术\cite{icc07xiao, globecom10xiao}等。

		WiFi安全的研究需要有开发和验证平台,用以实现和评估研究成果。
		相对于上层的安全研究,物理层的安全研究对验证平台提出很高的要求,
		802.11a/g协议中规定的物理层数据传输速率为54Mbps,延迟为16$\mu s$,软件实现的物理层无法在速率和延迟上满足要求,
		而硬件实现往往可编程性较差,开发和调试周期长,缺少可供参考的工具库。
		从近些年的研究看,验证平台的质量常常成为制约研究成果有效性的瓶颈,
		一些验证平台不支持物理层的编程,一些验证平台延迟过高无法与商用WiFi设备实时通信\cite{mobicom13securearray}。
		“工欲善其事,必先利其器”,支持物理层编程的高性能WiFi验证平台成为WiFi安全研究必不可少的工具。

		现有支持物理层WiFi研究的无线开放平台按物理层实现方式和编程方式主要分为四类:
		\begin{itemize}
			\item 计算机仿真软件,没有射频前端,物理层由纯软件实现,支持软件编程,有较多的参考样例,代表是Matlab;
			\item 商用网卡,有射频前端,物理层由ASIC实现,不支持软件编程但支持软件配置,代表是Intel 5300;
			\item 软件无线电平台,有射频前端,物理层由软件实现,支持软件编程,有较多的参考样例,代表是NI公司的硬件USRP与软件GNU Radio或LabVIEW的组合;
			\item 基于FPGA的无线电开放平台,有射频前端,物理层由FPGA实现,部分支持软件编程,参考样例少,代表是莱斯大学的WARP平台;
		\end{itemize}

		各类平台在可编程性和性能上各有优势,但目前没有平台在性能和可编程性上同时很好地满足近年来物理层WiFi安全研究,
		具体体现在没有平台在满足802.11协议规定的速率和延迟的同时,可以提供给研究者物理层的高可编程性。
		因此,物理层WiFi安全的研究者需要一个无线开放平台,在性能和可编程性上同时满足研究需求,用以实现和验证研究内容。

	\section{主要研究内容}
		本文的基本目标是为物理层WiFi安全的研究者提供一个无线验证平台GRTSEC(GRT for Security),
		围绕基本目标,本文的主要研究内容有以下几点:
		\begin{itemize}
			\item 对WiFi安全进行调研,总结常见的WiFi攻击以及WiFi安全相关的最新研究成果,分析WiFi安全研究对验证平台的需求;
			\item 对常见的WiFi验证平台进行调研,对各类平台进行对比和分析,找到各平台的优势以及不足;
			\item 设计适用于WiFi安全研究的框架,并基于FPGA进行实现;
			\item 为本文提出的验证平台设计使用样例,根据使用样例论证验证平台可以满足WiFi安全研究的需求;
			\item 使用本文提出的验证平台,搭建伪装WiFi,并提出一种新型的物理层WiFi识别技术识别不同WiFi设备。
		\end{itemize}

	\section{本文贡献}
		本文主要有以下贡献:
		\begin{itemize}
			\item 对物理层WiFi安全相关研究进行调研,分析了对WiFi验证平台的需求及现有系统的不足,
			定义了作为WiFi安全研究验证平台所需具有的特性;
			\item 在现有系统的基础上,设计并实现了一种支持物理层WiFi安全研究的验证平台,
			性能上满足与商用设备通信的要求,提供易于编程调用的API;
			% 通过使用样例论证了本文提出的验证平台可以满足绝大多数物理层WiFi安全研究的需求;
			\item 作为使用样例,提出了一种新型的物理层WiFi识别技术,可以对不同WiFi设备进行识别,
			利用本文提出的验证平台,在与商用设备实时通信中验证了其有效性。
		\end{itemize}

	\section{本文组织}
	第二章对WiFi安全相关的背景进行介绍;
	第三章对WiFi安全研究进行需求分析,提出了设计目标;
	第四章是验证平台的具体设计与其FPGA实现;
	第五章是验证平台使用样例的介绍,包括对已有研究的支持和新的安全认证技术;
	第六章是性能测试与评估,对平台的性能进行详细测试,对结果进行分析;
	第七章是文章总结与下一步研究的展望。

	% Copyright (c) 2008-2009 solvethis
% Copyright (c) 2010-2011 Casper Ti. Vector
% Public domain.

\chapter{背景介绍与相关工作}
本章将介绍WiFi的背景、WiFi安全的研究进展以及相关的WiFi验证平台。
本章首先在\ref{sec:comm_principle}节、\ref{sec:80211protocol}节介绍WiFi的背景知识,
然后在\ref{sec:security_research}节介绍物理层WiFi安全相关的研究,
其次在\ref{sec:related_work}节介绍现有的WiFi验证平台,
最后在\ref{sec:chap2_conclusion}节对本章进行小结。

	\section{WiFi物理层信息}\label{sec:comm_principle}
	研究者利用物理层信息加强WiFi的安全性,不同的物理层信息各有特点。
	本节介绍在WiFi安全研究中常见的几种物理层信息及其在无线通信中的物理意义。
		\subsection{RSS与RSSI}
		RSS(Received Signal Strength)是接收信号强度,指WiFi接收机收到的信号的功率,
		单位为dBm,为负数,越接近0表示信号质量越高。
		RSSI(Received Signal Strength Indicator)是接收信号强度指示,是无单位的。
		RSS一般为负值不太直观,人们把RSS人为映射为正值RSSI,理解和比较起来更直观,
		一般WiFi系统呈现给用户的是RSSI,不同系统有不同的映射方式\cite{wikirssi}。
		RSSI本质上与RSS只是同一个物理层信息的不同表示方法,文献里用二者都会出现,本文以RSSI进行表示。

		RSSI与无线发射机功率、收发距离、周围障碍物环境有关,发射机功率越大、收发距离越短,RSSI越大,而RSSI与障碍物环境关系比较复杂。
		不同的位置来源的信号一般RSSI也不同,因此无线通信的研究者常常通过RSSI获取相对位置。

		\subsection{CSI与CIR}\label{sec:background_csi}
		CSI(Channel State Information)是信道状态信息,是无线通信中从发射机到接收机之间信道的属性。
		在无线通信中,接收机接收到的信号与发送信号不完全相同,存在环境引起的变化,
		环境的影响称为信道,接收信号是发送信号与信道共同作用的结果,如式\ref{math:channel_model},
		\begin{equation}
			\centering
			r(t)=h(t)*s(t)+n
			\label{math:channel_model}
		\end{equation}
		$r(t)$代表接收信号,$h(t)$代表信道,$s(t)$代表发送信号,$n$代表噪声。

		接收机为了从接收信号中还原出发送信号,需要先得到信道信息,然后消除信道对接收信号的影响。
		接收机得到的信道信息,称作CSI。
		为了得到CSI,无线通信中常采用的方法是发射机发送一段双方已知的信号,称为训练序列,接收机对比接收信号和理论信号,
		得到信道信息,这个过程称为信道估计,从接收信号中消除信道影响的过程称为补偿\cite{book06commprin}。
		这里基于的一个假设是在短时间内信道变化不大,信道对训练序列的作用与信道对信号其它部分的作用相近。

		影响CSI的因素很多,有多径效应、多普勒效应、随距离的能量衰减等。
		多径效应是指由于周围物体的反射,信号经过多条路径到达接收机,每条路径效果叠加称为多径效应\cite{book06commprin}。
		多普勒效应是由发射机和接收机之间相对位置的变化引起。
		CSI具有丰富的物理内涵,常常被无线研究者用来推测其它信息,比如判断位置、判断是否有人经过、判断是否在移动。

		有的文献中也具体标明使用CIR这个物理量,CIR(Channel Impulse Response)是信道冲激响应,是一种特殊的CSI
		表示发送端发送一个脉冲信号的时候,接收端所能收到的信号值$h(t)$。
		CIR通常表示时域的信道特征,CSI既包含时域特征,也包含频域特征,是信道信息的总称。

		\subsection{频率偏移}
		介绍频率偏移之前,首先介绍频率同步。同步是指无线通信系统的接收机从收到的波形中找出有效信号,是无线通信中必不可少的步骤。
		同步分为两类,时间同步和频率同步。时间同步是指找到发送信号的起始时间,对于异步的通信系统都需要进行时间同步,如串口通信协议。
		接收信号和发送信号的载波频率存在频率偏移(简称频偏),频率同步是指找到并补偿频率偏移,还原出发送信号。
		频偏对频分复用系统会造成很严重的影响,因此频率同步是频分复用系统必要的组成部分。

		\begin{figure}[H]
			\centering
			\includegraphics[width=0.7\textwidth]{img/freq_offseet_channel_model.png}
			\caption{无线通信系统中频率偏移的信道模型}
			\label{fig:freq_offseet_channel_model}
		\end{figure}
		频偏的信道模型可以由图\ref{fig:freq_offseet_channel_model}表示,$s(t)$表示发送的基带信号,$r(t)$表示接收的基带信号,$h(t)$表示信道,
		混频器用来将基带信号与载波合成后搬移到射频上,图中省略了低噪声放大器等其他通信电路。
		如图\ref{fig:freq_offseet_channel_model}所示,频率偏移由接收机和发射机载波频率不一致引起,
		设发射机载波频率为$f$,接收机载波频率为$f+\Delta f$,$\Delta f$为频率偏移。
		频偏的数学模型可以表示为式\ref{math:freq_offset},噪声对频偏几乎没有影响,
		\begin{equation}
			\centering
			r(t)=h(t)*s(t) \cdot e^{-j2\pi ft} \cdot e^{j2\pi(f+\Delta f)t}
			\label{math:freq_offset}
		\end{equation}
		对于实际的无线通信系统,常常使用有周期性的训练序列进行频偏估计,根据周期的偏移推测频率偏移,然后进行频偏补偿。

	\section{802.11协议简介}\label{sec:80211protocol}
	WiFi通信的协议是IEEE 802.11协议集,规定了MAC(媒体访问控制)层和物理层的规范,
	具体协议按推出的时间顺序有802.11b、802.11a、802.11g、802.11n、802.11ac等。
	目前市面上的主流WiFi芯片都支持802.11a/g/n协议,本文主要基于802.11a/g/n协议进行阐述。
		\subsection{802.11物理层介绍}
		802.11物理层规定了数据传输的调制解调方式、编码方式、射频参数等,是OSI网络模型的最底层。
		除802.11b使用DSSS(direct-sequence spread spectrum,直接序列扩频)调制技术外,
		802.11a/g/n/ac等都使用了OFDM(Orthogonal frequency-division multiplexing,正交频分复用)调制技术。

		OFDM是一种将信号分配到多个子载波的频分调制方式,每一个子载波的调制后信号有部分重叠,但因为子载波之间严格正交,接收机可以分离出不同子载波的信号。
		OFDM的优点是在子载波间不需要保护间隔,频谱利用率高,且抗多径衰落能力强,
		缺点是对频率偏移敏感,需要对频偏进行纠正\cite{book06commprin}。
		OFDM由于其对复杂环境的抗干扰性,目前广泛应用于宽带通信,如WiFi、4G移动网络、数字电视、非对称数字用户环路(ADSL)等。
		802.11a/g协议规定的OFDM子载波数量为64个,其中传输数据的子载波是48个,传输导频的子载波是4个,
		802.11n协议20M带宽模式下子载波数量为64个,其中传输数据的子载波是52个,传输导频的子载波是4个,
		802.11n协议高吞吐率40M带宽模式下子载波数量为128个,其中传输数据的子载波是108个,传输导频的子载波是6个\cite{ieee80211}。

		\begin{figure}[H]
			\centering
			\includegraphics[width=0.7\textwidth]{img/80211_phy_module.png}
			\caption{802.11a/g物理层的简单模块示意图}
			\label{fig:80211_phy_module}
		\end{figure}
		图\ref{fig:80211_phy_module}是802.11a/g物理层的简单模块示意图,802.11n在此基础上增加对多天线的支持。
		发送端得到来自MAC层的数据,经过信源编码、星座映射、IFFT(Inverse Fast Fourier Transform,快速傅里叶逆变换)、插入保护间隔、插入前导码、射频前端到达空气中。
		其中,信源编码包括扰码、冗余编码、交织等步骤,目的是使信源数据更加强健;
		星座映射是将二进制数据映射到复数星座点上,有不同的映射策略,对应多种调制方式;
		IFFT是OFDM的关键步骤,将频域信号转变为时域信号,对于802.11a/g,此处是64点IFFT,对于802.11n,此处兼有64点IFFT和128点IFFT;
		插入保护间隔是指在时域的不同符号间插入一段间隔时间,降低码间串扰的影响;
		插入前导码是指在一帧前面增加训练字,如\ref{sec:comm_principle}节所述,训练字用来完成时间同步、频率同步、信道估计等功能。
		接收端从空气中得到无线信号,经过射频前端得到复数表示的I/Q两路信号,经过时间同步和频率同步后得到一帧,
		去掉保护间隔后经过FFT得到频域信号,经过信道估计和信道均衡后去除信道对信号的影响,
		经过解星座映射后由复数信号得到二进制数据,经过解码还原出真实数据,传给MAC层。

		802.11物理层中蕴含了很多对于WiFi安全研究有价值的信息,尤其是接收端模块,比如信道估计模块可以得到CSI,反映了发射机的位置、周边环境等,
		频率同步模块可以得到频率偏移,解星座映射模块可以得到接收信号的星座图。

		\subsection{802.11 MAC层介绍}
		在OSI网络模型中,物理层之上的一层是数据链路层,MAC层是数据链路层的子层,802.11协议规定的是物理层和MAC层的规范。
		MAC层提供了在不可靠的无线媒介中可靠传输用户数据的机制。本文主要面向物理层,会简单涉及MAC层,此处只对MAC层做简单介绍。

		802.11 MAC层有两种功能规范,必须具备的功能DCF(Distributed Coordination Function,分布式协调功能)
		和可选功能PCF(Point Coordination Function,集中式协调功能)。
		DCF主要基于CSMA/CA(Carrier sense multiple access with collision avoidance,带碰撞避免的载波侦听多址)技术。
		CSMA/CA技术是指一个无线节点在发送数据之前,首先侦听当前载波上是否有其他设备使用,只有没有其他人发送时可以发送数据。
		当有其他人发送数据时,首先等到空闲,然后再等待随机一段时间空闲后再发送,随机等待机制称为随机回退\cite{citsa05backoff}。

		\subsection{802.11安全机制}\label{subsec:80211security}
		802.11协议在MAC层规定了安全机制。
		1997年推出的最初的802.11协议将WEP(Wired Equivalent Privacy)认证方式应用于MAC层规范,
		后来在2003年,WiFi联盟提出WPA(Wi-Fi Protected Access)认证方式,
		在2004年,随着IEEE推出802.11i协议规定了WPA2认证方式,WEP认证被弃用\cite{wikiwep},
		在2012版802.11协议中,只保留WPA2认证方式,即802.11i协议\cite{ieee80211}。

		WEP和WPA被802.11协议弃用是因为其强度低易被破解,然而,WPA2认证方式也存在被攻击的可能。
		\cite{ijarcet12wpa2}中研究了WEP、WPA、WPA2的漏洞,并利用软件Aircrack-ng演示了如何破解这三种认证方式。
		MAC层安全机制的不完备性驱动越来越多的研究者开始探索物理层的安全机制,在第\ref{sec:security_research}节会介绍相关研究。

	\section{物理层WiFi安全研究进展}\label{sec:security_research}
	本节首先在\ref{subsec:attack_model}小节介绍WiFi攻击模型的研究进展,提出新的攻击模型是巩固WiFi安全的一种方式;
	其次介绍物理层WiFi安全机制的研究进展,研究者利用物理层信息进行用户的认证和设备的识别,
	物理层安全机制分为基于加密技术的安全机制和非加密安全机制,基于加密技术的安全机制在\ref{subsec:phy_tech_key_generation}小节进行介绍,
	非加密安全机制可以宽泛地分为硬件指纹技术和信道指纹技术\cite{ieeewc10noncryp}\cite{mobicom08radiometric},分别在
	\ref{subsec:phy_tech_device_fingerprint}、\ref{subsec:phy_tech_channel_fingerprint}小节进行介绍。
		\subsection{WiFi攻击模型的研究进展}\label{subsec:attack_model}
		相对于有线网络,无线网络的开放性给攻击者带来了可乘之机。针对WiFi的攻击按主动性可分为被动攻击和主动攻击。
		被动攻击主要是窃听获取密码,
		例如,通过监听CSI和上层网络报文,可以推测出用户在手机支付时输入的密码\cite{ccs16csi};
		通过窃听相邻设备的信道信息,获取基于信道的加密密钥\cite{ccs07robustkey}。
		主动攻击有欺骗攻击、DoS(Denial of Service,拒绝服务)攻击、密钥破解等,
		例如:攻击者伪装成合法设备MAC地址,只用10秒钟向AP发送去认证包,可使合法设备数分钟内无法连接AP\cite{usenix03dos};
		攻击者产生大量随机MAC地址向AP发送连接请求,消耗网络资源,可使DHCP服务器可分配IP被全部消耗\cite{wisec06spoofing};
		攻击者发起DoS攻击,不停地发送干扰信号,使合法设备之间无法正常通信\cite{milcom13jamming, jsac12antijamming};
		信号注入攻击可以破解密钥,攻击者通过向密钥达成一致的两台设备之间注入无线信号,
		使合法设备生成被自己控制的密钥,攻击者很大概率可以猜出密钥\cite{cns15key}。

		\subsection{物理层密钥技术}\label{subsec:phy_tech_key_generation}
		物理层密钥技术是指利用物理层信息生成密钥,对通信内容进行加密。
		对于密钥加密的无线通信系统,如何共享密钥是一个挑战,因为无线环境是开放的、公共的,攻击者可以轻易地监听到交换密钥的过程。
		无线通信中的密钥生成需要满足三个原则,时变性、互惠性、空间不相关性\cite{access16key}。
		时变性是指短时间内不变,一段时间后会发生变化,使密钥具有时效性;互惠性是指加密通信双方可以得到同一个密钥;空间不相关性指不同空间的设备得到不同的密钥。
		物理层的信道信息(RSSI或CSI)满足这三个原则,WiFi的通信双方之间共享同一个无线信道,会随时间变化,不同位置的信道不同,
		因此信道信息常常被用于密钥生成。

		例如,有文章讨论了基于RSS和CSI的密钥生成策略\cite{access16key},这篇文章研究了信道条件和信道参数对密钥生成的影响,分析得出RSS和CSI均可用于密钥生成,
		但基于RSS的密钥系统易受预测攻击,而基于CSI的不受此攻击影响。
		另有研究讨论了在信道互惠性无法满足时如何用CSI生成密钥\cite{infocom13key},
		提出使用信道增益补偿(Channel Gain Complement)的方法达成密钥一致。
		有文章提出向信道引入随机信号的密钥生成方法\cite{cns15key},合法设备A向合法设备B发送训练序列时,加入本地生成的一个均值为0的随机信号$X$,
		B收到时将信道值$H$乘以自己的随机信号$Y$,这样双方可以得到共同的密钥$XYH$,攻击者无法同时得到双方的随机信号。

		\subsection{硬件指纹技术}\label{subsec:phy_tech_device_fingerprint}
		硬件指纹技术是指利用硬件相关的物理层信息进行识别和认证的技术,这里的硬件主要指WiFi发射机和接收机的电路。
		在无线通信中,数据容易伪造,电路的固有属性难以模仿和复制,因此,硬件相关的物理层信息常常用来进行身份的识别。
		WiFi中硬件相关的物理层信息有时钟偏移、时钟波动、射频特性等。

		PARADIS系统\cite{mobicom08radiometric}利用多种物理层信息结合机器学习对802.11无线设备进行识别。
		硬件电路上的不完美性会引起射频信号的偏移,这篇文章收集与设备硬件关系紧密的物理层信息,
		包括频偏、训练字段与理论值的相关性、星座图中星座点的偏移,接收端提取这些特征与设备对应,作为训练集,由机器学习的方法对新收到的包进行判断。
		为了提高准确度,对多个连续的包取平均,这里与信道指纹技术结合,保证多个连续的包具有相同的信道指纹。
		这篇文章的实验表明,PARADIS系统可以识别超过130种商用网卡,准确度超过99\%。

		也有其他无线系统,如蓝牙,采用硬件指纹技术进行身份识别。
		例如BlueID系统\cite{infocom14blueid}利用蓝牙设备的时钟不同特性进行设备识别。

		\subsection{信道指纹技术}\label{subsec:phy_tech_channel_fingerprint}
		信道指纹技术是指利用信道相关的物理层信息进行识别和认证的技术,这里的信道指无线通信的信道,参见\ref{sec:background_csi}。
		在无线通信中,信道与调制方式、射频参数、周围环境等关系密切,
		因此,信道相关的物理层信息常常用来定位和判断设备是否被移动,在短时间内,也可以用来进行身份识别。
		信道相关的物理层信息有CSI、RSSI、多天线信号方向矩阵等。

		有研究利用RSSI对无线设备进行认证\cite{wisec06spoofing},具体来说,多个AP连接到一个称作WA(wireless appliance)的中心服务器上,
		AP将每一个成功接收的包的RSSI汇总给WA,由WA进行记录和认证,多个AP对同一个包记录的RSSI组成向量$S$,称作信号指纹(signalprint),
		$S_i$表示第$i$个AP记录的RSSI,若第$i$个AP没有收到此包,记录为默认最小值-95$dBm$。
		如果AP数量不足以在同一信道布置多台,则当发现有可疑包时(如去认证包)再将其他信道的AP调整到信道。
		此方法还可以用来识别同一台设备产生大量随机MAC引起的DoS攻击,因为同一台设备发出的包会拥有相近的信号指纹。
		这篇文章的方法有较大局限,比如多数实际场景不具备多台AP,待检测的设备不能移动,无法检测距离合法用户5米范围内的攻击者等。

		有研究提出利用CSI检测物联网设备是否是非法移动\cite{acsac15iot}。
		一些场景下物联网设备不希望被其他人移动,比如植物监控摄像头、办公室摄像头等。
		这篇文章利用CSI是否变化来判断设备是否被移动,需要与人员走动进行区别,
		具体方法是用多接收端区分环境变化和设备移动,如果是设备移动,会对所有接收端造成影响,如果是环境变化,不会对所有接收端影响。

		有研究提出利用多天线的CSI识别多种主动攻击\cite{mobicom13securearray}。
		多天线技术在802.11n协议开始使用,可以显著地提高系统传输速率和准确度。
		这篇文章利用多天线的CSI提取出信号到达角度,提出一套通过到达角度认证的DataCheck协议,对设备进行识别和认证。
		可以识别相隔5厘米的设备,相对于利用RSSI的识别距离5米\cite{wisec06spoofing},识别精度明显提高,
		而且多天线技术可以替代布置多设备,降低成本。

		\subsection{本节小结}
		本节首先介绍了常见的WiFi攻击,然后介绍了三种利用物理层信息应对WiFi攻击的策略,密钥技术、硬件指纹技术、信道指纹技术。
		其中密钥技术主要对802.11现有加密机制进行改进,提高传输内容的安全性,适合在商用802.11网卡上进行部署。
		硬件指纹技术和信道指纹技术都可以对设备进行身份验证,都是先学习再识别的模式,
		其中硬件指纹技术对学习过程的要求比较高,常常需要机器学习的技术,但可以应对设备移动的场景,
		信道指纹技术主要在短时间内对静止的设备进行识别。

	\section{现有的WiFi验证平台}\label{sec:related_work}
	在物理层WiFi安全的研究中,常见的验证平台主要分为四类:计算机仿真软件,商用网卡,软件无线电平台,基于FPGA的无线电开放平台。
	不同平台各有优劣,适用于不同的验证阶段和验证场景,下面将分别进行介绍。

		\subsection{计算机仿真软件}
		在无线通信中常用的计算机仿真软件有Matlab、WiSE等。
		Matlab是一款科学计算软件,由于其编程语言易于做通信算法中数据处理和图形化表示,被广泛应用于无线通信的算法仿真。
		在各大开源社区中,不乏各WiFi协议的开源实现,对于WiFi安全的研究者可以很方便地在此基础上加以改进,验证自己提出的新安全机制的有效性,
		一些研究是通过Matlab完成系统实现的\cite{globecom14key, icnc13cognitive, infocom14relay}。
		WiSE是Bell实验室在1995年推出的一套无线系统的仿真工具\cite{bell95wise},
		相比于Matlab,WiSE可以对室内空间进行建模,提供可视化的无线信号覆盖图,
		主要用于基站的布置,在较早的安全研究中也有用来做系统实现和验证的\cite{icc07xiao}。
		仿真软件的优势是可编程性高,具有可视化界面,但一个通信系统除了数据处理部分,还包括无线电前端和通信环境,这是仿真软件不具备的。
		另外,仿真软件使用通用处理器进行数据处理,速度相对专用无线网卡或可编程硬件电路要低。
		因此,对于大多数WiFi安全的研究,快速实现的软件仿真只是系统验证的第一步。
		总的来说,仿真软件的优点是易于编程,有众多可参考的开源代码,适合做理论上的快速验证,
		缺点是无法在真实的无线环境中进行验证,不能称作完整的通信系统。

		\subsection{商用网卡}
		相对于仿真软件,商用网卡可以在真实的无线环境中进行验证,但不是所有的商用网卡都支持物理层WiFi安全的研究,
		大多数商用网卡是用来上网的,而不是用来做开发验证。在WiFi安全的研究中,常用来做系统验证的有两类无线网卡,
		第一类是可提供物理层信息的网卡,例如Intel 5300\cite{ccs16csi, wisec14violating, acsac15iot},
		Soekris box\cite{mobicom13securearray},Atheros AR5212/AR2111\cite{mobicom08radiometric};
		第二类是支持开源无线网卡固件的网卡,例如LinkSys WRT54G\cite{wisec06spoofing}、WRT54GL\cite{iet13multilayer}支持OpenWrt。
		第二类网卡的数量较少,但开放度更高,功能多于第一类网卡,而且在开源社区有一些参考代码。
		商用网卡的缺点是不支持物理层的编程,因此在WiFi安全的研究中应用场景比较局限,无法自定义物理层帧结构和帧内容。
		总的来说,商用网卡的优点是可以在真实的实时无线环境中验证,常用于模拟WiFi攻击和进行物理层信息的分析,
		缺点是不具备物理层的可编程性,应用场景比较局限。

		\subsection{软件无线电平台}
		软件定义的无线电(Software-defined Radio)是指软件实现的无线电系统,软件无线电平台包括软件开发工具和硬件无线电外设,
		研究者用软件开发工具定义数据处理过程,再通过硬件无线电外设在真实的无线环境中进行通信。
		目前最常用的软件无线电平台是GNU Radio\cite{gnuradio}与USRP(Universal Software Radio Peripheral)\cite{ettus09usrp}的组合,
		GNU Radio作为图形化的软件开发工具,可与硬件无线电外设USRP连接,并提供了大量无线通信常用的模块。
		相对于仿真软件,软件无线电平台的优势是可在无线环境中通信,相对于商用网卡,软件无线电平台的优势是可自定义物理层数据处理流程,开放性高,
		可以说软件无线电平台是仿真软件和商用网卡的结合,在多项物理层WiFi安全的研究
		\cite{ccs16mumimo, cns15injection, globecom14location, tvt16spoofing}中作为验证平台。
		但软件无线电平台也有缺点,由于其物理层完全由软件定义,性能上无法与商用网卡媲美,
		速率、延时等关键指标甚至无法满足802.11协议的要求,无法与商用设备进行实时通信。

		\subsection{基于FPGA的无线电开放平台}
		FPGA(Field-Programmable Gate Array,现场可编程门阵列)作为可编程的硬件,近些年被应用于无线电的平台开发中。
		FPGA的特点是硬件逻辑全可定制,由硬件描述语言(hardware design language,HDL)对硬件进行编程,在具有高性能的同时兼备了可编程的能力。
		基于FPGA的无线电开放平台由硬件定义物理层,与无线电外设通过高速的接口连接,相比于无线网卡优点是提供了物理层的可编程能力,
		相比于软件无线电平台拥有更高的通信性能,代表是莱斯大学的WARP平台\cite{warp},一些研究\cite{mobicom13securearray, access16key}采用了此平台进行验证。
		基于FPGA的无线电开放平台目前处于发展阶段,还存在不少问题,以WARP平台最新版本WARPv3为例,有以下几点问题,
		\begin{itemize}
			\item 物理层可编程性较差,目前尚未有物理层WiFi安全的研究对物理层数据的硬件处理流程进行自定义;
			\item 延迟较高,不能与商用设备实时通信,有文章提到WARP的延迟无法在满足实时通信中密钥一致性的要求\cite{access16key},
			另有文章测试了使用WARP做验证的延迟为20毫秒左右,不满足协议中MAC层的时序要求\cite{mobicom13securearray};
			\item 相对于网络上层封闭,不能与上层协议栈相连,因此也不兼容常见的上层网络工具;
		\end{itemize}
		虽然基于FPGA的无线电开放平台存在不少问题,但在可编程性与通信性能结合方面最具前景,是目前平台开发的热点。
		本课题组(北京大学无线可重构体系结构课题小组\cite{pkuraw})先后提出GRT系统\cite{can14grt}和GRT2.0系统\cite{sigda17grt}。
		作为基于FPGA的无线电开放平台,GRT2.0系统的优点是在性能上可满足与商用设备实时通信的要求,可作为无线网卡使用,应用场景如全双工平台\cite{mna16grt}。
		本研究选择基于GRT2.0系统进行开发,提出支持物理层WiFi安全研究的验证平台GRTSEC(GRT for Security)。

		\subsection{本节小结}
		在本节中,我们介绍了物理层WiFi安全的研究中常用的四类验证平台。
		计算机仿真软件的编程性最好,有大量参考代码,但无法在真实的无线环境中进行验证;
		商用网卡的性能最好,可以在真实的无线环境中进行验证,但可编程性最差,不支持物理层编程;
		软件无线电平台具有很高的物理层编程性,可以进行无线通信,但通信性能较低,无法满足WiFi协议标准的要求;
		基于FPGA的无线电开放平台使用可编程的硬件实现物理层,兼具物理层的可编程性和高通信性能,尚处于发展阶段,存在一些问题。
		综合以上几点,本文选择基于GRT2.0系统进行开发,提出支持物理层WiFi安全研究的验证平台GRTSEC。

\section{本章小结}\label{sec:chap2_conclusion}
本章介绍了WiFi的背景、WiFi安全的研究进展以及相关的WiFi验证平台。
在\ref{sec:comm_principle}节介绍了物理层信息及其在无线通信中的物理意义,
在\ref{sec:80211protocol}节介绍了802.11协议,
在\ref{sec:security_research}节介绍了物理层WiFi安全相关的研究,
在\ref{sec:related_work}节介绍了现有的WiFi验证平台,作为相关工作。
本文选择基于北京大学的GRT2.0平台进行开发,提出支持物理层WiFi安全研究的验证平台GRTSEC。

	% Copyright (c) 2008-2009 solvethis
% Copyright (c) 2010-2011 Casper Ti. Vector
% Public domain.

\chapter{需求分析与设计目标}\label{chap:demand}
本章将分析物理层WiFi安全研究对验证平台的需求,提出本文的设计目标。
本章首先在\ref{sec:platform_demand}节分析物理层WiFi安全研究对验证平台的需求,
然后在\ref{sec:design_goal}节提出本文的设计目标。

	\section{安全研究对验证平台的需求}\label{sec:platform_demand}
	本节分析物理层WiFi安全研究对验证平台的需求。
	首先分析模拟WiFi攻击的需求,见\ref{subsec:demand_perform_attack}小节,
	然后研究安全机制的需求,分为基于加密和基于非加密两类,基于加密的安全机制见\ref{subsec:demand_encryption}小节,
	基于非加密的安全机制主要是基于身份验证,见\ref{subsec:demand_identification}小节。

		\subsection{模拟WiFi攻击}\label{subsec:demand_perform_attack}
		\ref{sec:security_research}节曾介绍过,WiFi攻击有多种类型,伪装、DoS攻击、密钥破解等。
		模拟不同的WiFi攻击对验证平台有一些共同的需求,可以总结为以下几点。
		\begin{itemize}
			\item 可监听报文,可兼容上层网络分析工具。
			监听报文是网络设备的基本功能,而上层网络分析工具在发起WiFi攻击时很常用,例如使用网络工具Aircrack-ng破解WEP、WPA、WPA2加密\cite{ijarcet12wpa2}。
			\item 接收端提供常见的物理层信息,如CSI、RSSI等。
			一些攻击者除了需要监听合法设备报文内容外,还需要获取接收信号的物理层信息,用来推测用户所发内容\cite{ccs16csi}或破解密钥\cite{cns15key}等。
			\item 发送端任意修改MAC层帧内容,包括IP地址、MAC地址、AP模式下的SSID等。
			自定义帧内容是伪装攻击的前提,如需要修改源MAC地址和目的MAC地址发起DoS攻击\cite{wisec06spoofing};
			\item 可与商用设备实时通信。攻击对象是商用设备,如果验证平台不能与商用设备实时通信,则可发起的攻击会很受限。
		\end{itemize}

		\subsection{基于加密的安全机制}\label{subsec:demand_encryption}
		对传输内容进行加密是网络安全中常采取的措施,攻击者虽然可以“听”到内容但却无法理解。
		基于加密的安全机制分成几个关键步骤,密钥生成、密钥分发、加密解密。
		在\ref{subsec:80211security}小节提到,现有的WiFi加密协议有漏洞,研究者利用物理层信息对加密协议进行加强,对验证平台提出以下需求。
		\begin{itemize}
			\item 接收端提供常见的物理层信息,如CSI、RSSI等。
			在基于加密的WiFi安全研究中或利用CSI生成密钥\cite{infocom13key}、\cite{iwqos14key},
			或利用RSSI生成密钥\cite{mobicom08key}、\cite{mobicom09key}、\cite{tmc13key},
			也有文章研究了CSI和RSSI生成密钥的性能\cite{access16key}。
			表\ref{tab:key_generation_summary}总结了现有密钥生成的研究所需的物理层信息。
			\item 可兼容上层网络分析工具。有研究需要使用wireshark进行抓包\cite{tmc13key}。
			\item 可在实时通信中验证,通信双方达成密钥一致性。有研究使用WARP平台进行验证\cite{access16key},
			但由于WARP平台延迟高,无法满足密钥一致性的要求。
			\item 支持软硬件加密解密模块替换。加密解密过程需要密集的计算,软件可以快速实现加密算法,硬件可以提高效率,满足协议要求的时序。
		\end{itemize}
		\begin{table}[!hbp]
		\centering
		\caption{现有基于物理层信息的密钥生成系统总结}
		\label{tab:key_generation_summary}
			\begin{tabular}{|l|l|l|l|} \hline
			现有工作 & 物理层信息 & 具体技术 & 使用的验证平台 \\ \hline
			INFOCOM 13\cite{infocom13key} & CSI & 加时间戳,信道补偿 & Intel 5300 NIC \\ \hline
			IWQoS 14\cite{iwqos14key} & CSI & 联合多子载波,哈希 & Intel 5300 NIC \\ \hline
			TMC 13\cite{tmc13key} & RSSI & 信息重构 & Atheros AR 5B95 \\ \hline
			Access 16\cite{access16key} & RSSI CSI & 多环境测试分析结果 & WARPv3 \\ \hline
			\end{tabular}
		\end{table}

		\subsection{基于身份验证的安全机制}\label{subsec:demand_identification}
		基于加密的安全机制有一定局限性,复杂的加密算法难以在实际通信系统中实现\cite{ccs16mumimo},
		因此一些研究者探索非加密的安全机制,即基于身份验证的安全机制\cite{ieeewc10noncryp}。
		对验证平台提出以下需求。
		\begin{itemize}
			\item 接收端提供常见的物理层信息,如CSI、RSSI等,除此之外,还支持扩展物理层信息。
			例如使用频偏、信号相关性和星座点偏移\cite{mobicom08radiometric},
			使用CIR\cite{milcom11cir}。
			表\ref{tab:phy_auth_summary}总结了现有物理层身份验证的研究所需的物理层信息。
			\item 自定义物理层数据处理流程。自定义物理层帧格式和物理层控制逻辑,
			例如有研究在物理层发送端引入了人为的频偏\cite{asiaccs14csi},只有支持物理层编程的平台可以做到这一点。
			\item 可与商用设备实时通信,增加验证结果的可信度。
			\item 支持软硬件物理层信息处理模块替换。对物理层信息的处理过程需要密集的计算,软件可以快速实现加密算法,硬件可以提高效率,满足协议要求的时序。
		\end{itemize}
		\begin{table}[!hbp]
		\centering
		\caption{现有基于物理层信息的身份验证系统总结}
		\label{tab:phy_auth_summary}
			\begin{tabular}{|p{0.2\textwidth}|l|p{0.2\textwidth}|p{0.3\textwidth}|} \hline
			现有工作 & 物理层信息 & 具体技术 & 使用的验证平台 \\ \hline
			IEEE TPDS 13\cite{tpds13spoofing} & RSS & 聚类分析,支持向量机 & Orinoco silver card,Atheros miniPCI NIC \\ \hline
			IEEE Wireless Commun 10\cite{ieeewc10noncryp} & RSS & RSS序列反馈 & Dell Latitude E5400 laptop \\ \hline
			TVT 16\cite{tvt16spoofing} & RSSI & 博弈论,增强学习 & USRP N210 \\ \hline
			MobiSys 10\cite{mobisys10proximity} & RSS & 临近认证 & Nokia N800 Internet Tablets \\ \hline
			MILCOM 11\cite{milcom11cir} & CIR & 噪声消除 & 仿真软件 \\ \hline
			ICC 13\cite{icc13cir} & CIR & 利用多径延迟 & 仿真软件 \\ \hline
			INFOCOM 13\cite{infocom13csi} & CSI & CSI精度增强 & Intel IWL5300 NIC \\ \hline
			ASIA CCS 14\cite{asiaccs14csi} & 频偏 & 引入人为频偏 & USRP \\ \hline
			ICC 07\cite{icc07xiao} & 频偏 & 多径效应 & 仿真软件WiSE\cite{bell95wise} \\ \hline
			TMC 10\cite{tmc10clock} & 时钟偏移 & 时间戳 & Linksys WPC 55AG, Intel 3945ABG \\ \hline
			\end{tabular}
		\end{table}

	\section{设计目标}\label{sec:design_goal}
	综合以上模拟WiFi攻击、基于加密的安全机制、基于身份验证的安全机制对验证平台的需求,
	为了实现一种支持物理层WiFi安全研究的验证平台,本文对验证平台提出了如下设计目标:
	\begin{itemize}
		\item 实现802.11协议,可与商用无线网卡实时通信;
		\item 支持自定义数据处理流程,包括物理层和MAC层;
		\item 方便获取常用物理层信息,如RSSI、CSI、频偏等,支持用户扩展物理层信息;
		\item 支持安全算法的软件和硬件实现,提供软硬件模块互换的能力;
		\item 可与上层网络协议栈连接,兼容常见网络分析工具。
	\end{itemize}

	实现上述目标一个挑战是如何保证性能的同时,提供物理层足够高的可编程性。
	一般而言,可编程性较高的平台性能较差,比如第二章提到的GNU Radio,使用C++和Python编程,
	提供方便的可视化编程界面,但吞吐率和延时无法达到802.11a/g的要求。
	性能满足要求的平台一般可编程性差,甚至不具备物理层的可编程能力,
	商用网卡的性能最高,可以完全满足协议要求,但不具备物理层的可编程性。
	本文基于北京大学的GRT系统\cite{can14grt}进行开发,提出支持物理层WiFi安全研究的验证平台GRTSEC(GRT for Security)。
	GRT系统同时也是本文的前期工作,在第\ref{chap:grt2.0}章会部分进行介绍。

	% Copyright (c) 2008-2009 solvethis
% Copyright (c) 2010-2011 Casper Ti. Vector
% Public domain.

\chapter{GRT2.0系统及其射频通信库设计与实现}\label{chap:grt2.0}
	GRT2.0系统是本研究的前期工作,是本课题组合作完成的一款高性能可重构的软件无线电通信平台\cite{pkuraw},
	旨在帮助无线通信的软硬件研究与开发人员更好更快地在真实的通信系统中实现和验证无线通信系统物理层和MAC层的算法,
	GRT2.0系统是其最新的发布版本,应用有全双工\cite{mna16grt}、认知无线电\cite{fpga17grt}、低延迟通信等。
	本人负责其中射频通信库和工程自动化脚本,参与了LOW MAC模块的搭建。
	本节将对整体进行简单介绍,对本人完成的射频通信库的设计与实现进行具体展开。
	在\ref{sec:grt2.0_overview}小节介绍GRT2.0系统的整体框架,
	在\ref{sec:grt2.0_rfd}小节介绍射频通信库的设计,
	在\ref{sec:grt2.0_script}小节介绍工程自动化脚本,
	在\ref{sec:grt2.0_drawback}小节分析GRT2.0系统用于物理层WiFi安全研究时的不足,
	在\ref{sec:grt2.0_summary}小节对本节进行简单总结。

	\section{GRT2.0整体框架介绍}\label{sec:grt2.0_overview}
	在硬件组成上,GRT2.0系统包括四个部分,FPGA、上位主机、FPGA配置计算机、射频前端,
	如图\ref{fig:grt_overview}所示,
	其中上位主机与FPGA配置计算机可以是同一台计算机,各部分介绍如下:
		\begin{itemize}
			\item FPGA:作为无线电通信平台的核心数据处理硬件,包括射频通信库、物理层模块、LOW MAC模块、USB通信库四个部分,
			射频通信库负责射频前端的交互,
			物理层模块实现了802.11规定的物理层全部的数据处理过程,
			LOW MAC模块实现了802.11规定的MAC层的时序要求高的数据处理过程和控制流程,
			USB通信库负责与上位计算机的交互。
			\item 上位主机:运行GRT2.0的驱动,与上层网络协议栈连接,运行应用程序。
			\item FPGA配置计算机:FPGA的开发、烧写与调试。
			\item 射频前端:负责射频通信。
		\end{itemize}

		\begin{figure}
			\centering
			\includegraphics[width=1.0\textwidth]{img/GRT_overview.png}
			\caption{GRT2.0整体结构示意图}
			\label{fig:grt_overview}
		\end{figure}
	目前GRT2.0系统的FPGA部分使用的是Xilinx KC705开发板,后续版本也对Xilinx VC707进行了支持,
	上位主机使用的操作系统是Ubuntu14.04,
	FPGA配置计算机使用的开发工具是Vivado2015.2及其对应版本的SDK,
	射频前端使用的是Analog Device公司的EVAL-AD-FMCOMMS3-EBZ开发板\cite{fmcomms3}。

	值得一提的是,FPGA上的LOW MAC模块采用了软硬件协作的结构,
	将嵌入式处理器MicroBlaze\cite{microblaze}与硬件IP核\cite{xilinxip}结合,
	图\ref{fig:grt_lowmac}是LOW MAC模块软硬件协作的结构示意图,我参与了LOW MAC模块的搭建。
	硬件IP核是对完成特定功能的硬件逻辑的封装,
	嵌入式软件与硬件IP核结合的方式可以大大提高系统编程的灵活性。
	流程控制在嵌入式软件中实现,例如根据一帧的MAC地址进行分支处理,
	如果由硬件实现会非常繁琐,可读性和可扩展性差,
	数据通路和密集计算在硬件IP核中实现,例如发送端增加CRC校验位、接收端检测CRC是否匹配,
	硬件IP核可以每个时钟周期计算64bit的CRC,
	如果由软件实现会造成很高的延迟。
	在Vivado开发套件中,嵌入式处理器MicroBlaze与硬件IP核共同组织成block design\cite{xilinxblockdesign}的形式,具有图形化的开发界面。
	为了提高可编程性,本文在GRTSEC的设计中,也加入了软硬件协作的结构,第\ref{chap:grtsec_design}章会进行进一步介绍。

		\begin{figure}
			\centering
			\includegraphics[width=1.0\textwidth]{img/GRT_lowmac.png}
			\caption{LOW MAC模块软硬件协同结构示意图}
			\label{fig:grt_lowmac}
		\end{figure}

	\section{射频通信库}\label{sec:grt2.0_rfd}
	射频前端是无线开放平台必备的组成部分,负责将数字基带信号转变为射频信号发送到空气中,
	以及接受空气中的射频信号转变为数字基带信号,射频前端的通信性能决定了无线平台的适用场景,
	对于WiFi无线验证平台,需要一个支持2.4GHz和5GHz中心频点、40MHz带宽的射频前端。
	作为一个提供给研究者的开放平台,除了考虑通信性能外,还要考虑价格,
	价格过高会让研究者望而却步。例如,OpenMili平台使用的射频前端\cite{mobicom16openmili}价格约11000美元,用于无线研究时价格过高。

	在GRT1.0系统中,我们选择使用USRP N210搭配XCVR2450子板\cite{usrpn210}作为射频前端,
	USRP N210价格是17000元,XCVR2450子板价格是4000元,属于可接受的范围。
	USRP的驱动是运行在Linux操作系统上的,称作UHD(USRP Hardware Driver),
	为了将FPGA开发板与射频前端相连,我们在FPGA上实现了USRP的硬件驱动,作为GRT1.0的射频通信库。
	这样的方式有以下几个缺点:
		\begin{itemize}
			\item XCVR2450子板最高支持25MHz带宽,无法达到802.11n协议要求的40MHz;
			\item USRP N210与FPGA开发板通过千兆以太网接口相连,
			以太网协议栈引入了不必要的延迟,经过GRT1.0射频通信库优化后的延迟也高达30ms以上,
			超过了802.11协议中要求的18ms的回复ACK的间隔,
			虽然可以与某些延迟要求宽松的商用设备实时通信,但系统的扩展性变得很差;
			\item 硬件实现USRP驱动的过程复杂,无法利用官方提供的软件驱动程序,也不方便驱动程序的升级换代;
			\item 不支持多天线MIMO(Multiple Input Multiple Output),MIMO技术是802.11n协议要求的技术。
		\end{itemize}

	以上缺点制约了无线平台的应用范围。

	在GRT2.0中,经过多方比较,我们选择使用性能更好、支持2x2 MIMO、具有通用接口的射频前端,
	Analog Device公司的EVAL-AD-FMCOMMS3-EBZ射频开发板\cite{fmcomms3},简称FMCOMMS3。
	FMCOMMS3的价格为6200元,射频芯片为Analog Device公司的AD9361,
	支持的中心频率范围是70MHz至6GHz,可以覆盖WiFi使用的2.4GHz和5GHz频段,
	带宽为200kHz至56MHz可调,支持802.11a协议要求的20MHz带宽、
	802.11b协议要求的18MHz带宽和802.11n协议要求的40MHz带宽,支持2x2 MIMO,
	通过高带宽的FMC接口(FPGA Mezzanine Card)\cite{wikifmc}与FPGA相连。
	虽然生产商提供了FPGA参考设计,但仍然无法直接与GRT系统集成,例如占用了过多的FPGA资源,没有提供上位主机配置射频参数的接口。
	射频通信库完成了GRT2.0系统与FMCOMMS3射频子板的对接。

	GRT2.0射频通信库的设计采用了嵌入式处理器与硬件IP核结合的方式。
	射频通信库软件部分所做的工作主要是对配置射频参数的接口进行封装,
	通过USB通信库与上位主机的驱动程序连接起来。
	由主机驱动程序配置的射频参数有中心频率和带宽,由射频通信库软件接口提供给主机驱动程序的是RSSI。

	射频通信库硬件部分对生产商提供的参考设计进行了较大的改造,体现在以下几点:
		\begin{itemize}
			\item 参考设计通过以太网与主机相连,通过HDMI与显示器相连,
			为降低FPGA资源使用率,射频通信库硬件部分去除对多种冗余接口的控制器,包括与主机相连的以太网接口控制器和以太网DMA控制器,
			与显示器相连的HDMI接口控制器,与板上LCD显示屏相连的LCD控制器,IIC接口控制器,只保留与FMC子板相连的FMC接口控制器;
			\item 参考设计将主机通过以太网发来的数据DMA到DDR3中,然后从DDR3中读取数据,
			经过AD9361 IP核转为时钟对齐的IQ两路数据,发送给FMCOMMS3子板,接收端与之方向相反,
			为降低数据传输延迟,射频通信库硬件部分去除收发数据读写DDR3的过程,直接转为FIFO接口与GRT系统的物理层模块相连;
			\item 参考设计中MicroBlaze配置级别高,性能好,
			为降低MicroBlaze占用的FPGA资源,射频通信库硬件部分简化了MicroBlaze的配置,
			提供支持射频通信库软件部分和LOW MAC软件部分的最小集合。
		\end{itemize}

	下面分别介绍射频通信库的数据通路的硬件接口和配置的嵌入式软件接口。
	在硬件设计中,射频通信库将AD9361 IP核输出的IQ两路数据转化为标准的FIFO接口,提供给物理层,并通过异步FIFO的方式完成时钟域转换。
	射频侧的时钟和复位信号由AD9361 IP核提供,这个时钟是随采样率变化的;物理层侧的时钟和复位信号由物理层提供。
	硬件接口以简单明了为目标,
	射频通信库与物理层之间的接口定义如下:
	\begin{lstlisting}[language={Verilog}]
module fmc_iface
(
input fmc_clk,
input fmc_rst,
input PHY_TX_clk,
input PHY_RX_clk,
input PHY_rst,
output PHY2RF_FIFO_prog_full,
input PHY2RF_FIFO_wr_en,
input [31:0] PHY2RF_FIFO_din,
input RF2PHY_FIFO_rd_en,
output [31:0] RF2PHY_FIFO_dout,
output RF2PHY_FIFO_empty,
output RF2PHY_FIFO_almost_empty
);
	\end{lstlisting}

	射频通信库在嵌入式软件端提供了配置射频参数接口。
	包括初始化射频子板,对射频子板的ADC、DAC、滤波器等初始化配置,设置中心频率、采样率、发送增益等射频参数。
	中心频率的单位是MHz,比如WiFi标准频段的信道1为2412MHz。
	采样率的单位为Sps(Samples per seconds),比如802.11a/g的采样率为20000000Sps,802.11n需要40000000Sps的模式。
	发送增益以衰减的形式表示,单位是mdb,即db的千分之一,设置为0时表示无衰减,1000时为衰减为十分之一。
	射频通信库提供的软件配置接口如下:
	\begin{lstlisting}[language={C}]
int fmc_main(void);
void set_rx_samp_freq(double* param, char param_no);
void set_tx_samp_freq(double* param, char param_no);
void set_rx_lo_freq(double* param, char param_no);
void set_tx_lo_freq(double* param, char param_no);
void set_tx1_attenuation(double* param, char param_no);
	\end{lstlisting}

	\section{工程自动化脚本}\label{sec:grt2.0_script}
	TCL脚本在FPGA的开发过程中有着广泛的运用\cite{xilinxtcl},从模块间引脚的连接、工程文件的组织,
	到调试信息的导出、自动执行工作流程,TCL脚本大大提高了FPGA开发的效率。
	为了将GRT2.0系统开源给研究者,我们提供了可以自动搭建工程项目的TCL脚本,
	实现了从源代码、资源文件到最终二进制文件的自动化过程。
	TCL脚本的使用是从GRT1.0到GRT2.0的一个重要改进。

	GRT2.0系统提供的工程自动化TCL脚本完成了以下过程:
		\begin{itemize}
			\item 对GRT2.0提供的完成特定功能的硬件逻辑封装成IP核;
			\item 生成block design中的各个模块和输入输出端口,完成模块间、端口间互相的连线;
			\item 生成和配置嵌入式处理器MicroBlaze及其附属的总线控制器,对总线地址进行分配;
			\item 对Xilinx提供的IP核进行配置,并添加到工程中,主要是各种不同类型的FIFO、时钟生成器、RAM、DDR3控制器等;
			\item 添加GRT2.0的资源文件,有约束文件、三角函数数据文件、OFDM导频数据文件等;
			\item 执行工作流程,设置流程策略,工作流程有综合(Synthesis)、布局(Place)、布线(Route)、生成二进制文件(Generate Bitstream)等。
		\end{itemize}

	通过执行工程自动化TCL脚本,用户可以利用手中的开发板和开源的GRT2.0代码,生成可以烧写FPGA的二进制文件,
	再结合我们提供的上位主机的驱动程序,用户便自行搭建GRT2.0系统完成。

	\section{用于安全研究时的不足}\label{sec:grt2.0_drawback}
	基于GRT2.0系统进行开发,可以满足设计目标中的与商用网卡实时通信,以及与上层网络协议栈连接,
	但在其他方面存在不足,尤其是无法获取物理层WiFi安全研究所需要的物理层信息。
	GRT2.0系统中物理层接收端的设计目标是消除接收信号的偏差,还原出发送信号,
	频偏纠正、相位纠正、信道均衡、Viterbi解码等过程都是为了完成这个目标,
	然而在物理层WiFi安全的研究中,接收信号的偏差有其背后的物理含义,常常被拿来做研究,
	但GRT2.0的系统框架不支持这些信息的提取。
	具体来说,用于安全研究时的不足包括以下几点。
		\begin{itemize}
			\item 物理层由单一的硬件逻辑实现,可编程性还不够高;
			\item 未提供方便获取CSI、RSSI的接口,CSI和RSSI是安全研究常用的指标;
			\item 在接收端数据处理过程中消除了信号偏差,而这些偏差在安全研究中常常被用到,例如频偏;
			\item 不支持软硬件数据处理模块替换的能力;
			\item 缺乏物理层编程的使用样例。
		\end{itemize}

	在第\ref{chap:grtsec_design}章将针对以上问题加以改进,
	介绍适用于物理层WiFi安全研究的验证平台GRTSEC的设计与实现。

	\section{本章小结}\label{sec:grt2.0_summary}
	本节对支持物理层WiFi安全研究的验证平台GRTSEC的前期工作GRT2.0系统进行了介绍,
	GRT2.0系统硬件上包括上位主机、FPGA、射频前端、配置计算机,
	由物理层模块、LOW MAC模块、射频通信库、USB通信库等模块组成。
	GRT2.0系统可以满足本文设计目标中的与商用网卡实时通信,以及与上层网络协议栈连接,
	但在获取物理层信息、物理层编程性、软硬件模块替换等方面存在不足。
	另外,本节展开介绍了本人在GRT2.0系统中完成的工作,主要负责完成的射频通信库、工程自动化脚本,
	参与完成的LOW MAC软硬件协作设计。

	% Copyright (c) 2008-2009 solvethis
% Copyright (c) 2010-2011 Casper Ti. Vector
% Public domain.

\chapter{GRTSEC设计与实现}\label{chap:grtsec_design}
	在本节将介绍GRTSEC在GRT2.0系统上的改进,
	\ref{sec:grtsec_overview}小节对总体设计和设计思路进行介绍,
	GRTSEC最核心的改进是对物理层信息进行了支持,提取了物理层信息并提供了物理层信息的编程接口,
	\ref{sec:grtsec_phyinfo_substract}小节介绍物理层信息的提取方法,
	\ref{sec:grtsec_phyinfo_api}小节介绍物理层信息的软件编程接口,
	\ref{sec:grtsec_phyinfo_analyzer}小节介绍用于做物理层信息分析的硬件模块设计,
	\ref{sec:grtsec_phyinfo_extension}小节介绍用户如何得到自定义的物理层信息,
	\ref{sec:grtsec_loopback}小节介绍为提高开发调试的效率,射频通信库模块做的改进。

	\section{总体设计}\label{sec:grtsec_overview}
	GRT2.0在用于WiFi安全研究时主要的问题是嵌入式软件与物理层硬件的交互不便,无法获取到物理层信息,
	也无法对物理层模块进行配置和数据传输。
	针对这个问题,我们在GRTSEC的总体设计上提出以下改进,体现在三方面:
	一是扩展了软硬件协同工作的范围,将物理层也与嵌入式处理器通过总线相连,
	方便用户直接对物理层进行配置,以及从物理层获取数据;
	二是提取了常用的物理层信息并提供给用户软硬件的编程接口;
	三是设计了对物理层信息进行分析的软件代码和硬件模块。
	这样做的好处是WiFi安全的研究者可以方便地从软件端获取常见物理层信息,以及扩展其它的物理层信息,
	并且可以选择由硬件或软件实现自己提出的安全机制,验证其正确性。

	GRTSEC的硬件结构如图\ref{fig:grtsec_hw_design},
	图中block design内为软硬件协同设计的部分,由于暂未有对USB通信库编程的需求,放在block design之外,
	图中Core为原GRT2.0中的嵌入式MicroBlaze处理器,
	Low mac TX、Low mac RX为原GRT2.0中LOW MAC模块的发送端和接收端部分,
	以上为保留的GRT2.0的设计。
	图中PHY TX和PHY RX是将原先的物理层模块移入block design中,与嵌入式处理器通过AXI总线相连,
	射频通信库也移入block design中,
	图中PHY Analyzer模式是GRTSEC新增加的硬件IP,用于对从物理层模块得到的物理层信息进行数据分析,
	并将分析结果反馈给嵌入式软件。
		\begin{figure}
			\centering
			\includegraphics[width=1.0\textwidth]{img/GRTSEC_hw_design.png}
			\caption{GRTSEC硬件模块设计图}
			\label{fig:grtsec_hw_design}
		\end{figure}


	\section{常用物理层信息的提取}\label{sec:grtsec_phyinfo_substract}
	% 在实际的物理层信息提取过程中,我们对物理层信息进行分类,
	% 第一类是随帧物理层信息,每一帧包含一组值,通过前导码(preamble)计算得到,比如CSI、频偏、RSSI、调制方式,
	% 第二类是随符号物理层信息,每一个符号(symbol)包含一组值,每一帧有多个符号,最典型的是导频,
	% 第三类是随采样点物理层信息,每个采样点有一个值,比如星座点偏移向量。
	% 在物理层WiFi安全的研究中,绝大多数使用的是第一类,随帧物理层信息,GRTSEC对这一类物理层信息进行了提取。
	% GRTSEC虽然未提供随符号物理层信息和随采样点物理层信息的接口,但在各个模块中包含有原始信息,用户可自行提取,
	% 参照\ref{sec:grtsec_phyinfo_extension}小节进行扩展。

	在实际的物理层信息提取过程中,物理层信息如CSI、频偏、RSSI、调制方式等,往往是跟随一帧的,
	即接收到的一帧对应了一组物理层信息,这些物理层信息也大多是通过一帧的前导码(preamble)计算得到。
	对于物理层信息的提取,我们面临着两个问题,一是物理层信息的帧对齐问题,二是精确接收时间问题。
	物理层信息的帧对齐问题是指当软件MAC层收到一帧时,如何获取该帧对应的物理层信息,
	假如此时去物理层取,有可能下一帧已经到来,得到的是下一帧的信息,也有可能当前帧的物理层信息尚未计算完毕,得到的是上一帧的信息。
	精确接收时间问题是指软件MAC层收到一帧时如何知道该帧的精确到来时间,
	在TMC 10\cite{tmc10clock}这篇文章中提出利用帧到来时间作为计算硬件指纹的依据,到来时间必须精确到us级别,
	但软件代码的执行会受操作系统调度的影响,即使是嵌入式软件,也会受中断的影响,无法精确的标定帧到来的时间。

	对于物理层信息的帧对齐问题,我们采用三级缓存的方法。
	当物理层收到一帧,各个模块通过前导码计算得到物理层信息后,保存在模块内部,与模块的时钟对齐,称作模块内缓存。
	SIGNAL字段在前导码的后面,当此帧的SIGNAL字段被正确解出,且LOW MAC准备好接收下一帧时,物理层会通知LOW MAC模块收到了一帧,
	向嵌入式处理器发起一次中断,此时,我们将模块内缓存的物理层信息,经过时钟域转换,在另一组统一时钟的寄存器中缓存,称作物理层缓存,
	这一步的目的是让各个模块可以继续处理下一帧,更新模块内缓存。
	LOW MAC的软件收到中断后,将物理层缓存的物理层信息,存放到与AXI总线时钟对齐的第三组寄存器中,称作AXI缓存,
	处理器通过AXI总线读取AXI缓存的物理层信息,此时读到的即为收到中断的那一帧的物理层信息,
	这一步的目的是让物理层可以向LOW MAC发起下一帧的终端,而嵌入式处理器在一帧之内的后面的任意位置都可以读取这一帧的物理层信息。
	我们将物理层信息缓存在寄存器中,而不是将物理层信息缓存在队列中,因为SIGNAL有可能会解错,嵌入式软件也可能会丢掉中断,
	一旦SIGNAL错误或中断没有被处理,缓存的物理层信息就不会继续被读取,阻塞了队列,队列发生错位。
	假如缓存在寄存器中,发生错误时后面帧的物理层信息会覆盖前面的,不会引起错位。
	图\ref{fig:phy_info_cache}描述了三级缓存的思想。
		\begin{figure}
			\centering
			\includegraphics[width=0.7\textwidth]{img/phy_info_cache.png}
			\caption{GRTSEC物理层信息三级缓存示意图}
			\label{fig:phy_info_cache}
		\end{figure}

	采用三级缓存方法的好处是,在保证一帧的数据段与物理层信息保持同步的同时,不阻塞后面帧的接收和数据处理过程,
	并且使用寄存器而不是FIFO,节省了存储空间。

	对于精确接收问题,GRTSEC采用物理层时间戳的方法。首先,我们在物理层与LOW MAC接口处使用一个64位的计数器,每一个时钟周期加一,
	目前此处的时钟频率是100MHz,那么计数器的精度就是10ns。然后,当物理层同步到一帧时,读取计数器的值并保持下来,
	LOW MAC软件通过AXI总线读取保存的计数器值。这种方法是把计数器的值也当做一种物理层信息,采用三级缓存技术进行记录,计数器相当于模块内缓存。
	AXI总线一次读写的宽度是32位,一般而言,读取计数器的低32位寄存器即可,低32位可以记录约43秒的时间间隔,对于超过43秒的时间,则读取高32位寄存器。

	采用物理层时间戳的好处是,用户可以精确地获悉当前帧的相对接收时间,一方面可以单独作为一个物理层信息计算得到时钟偏移\cite{tmc10clock},
	另一方面可以与其他物理层信息进行组合,得到其他物理层信息随时间变化的关系,物理层信息的时变性也具有重要的物理含义。

	接下来介绍几种常用随帧物理层信息的提取方法,具体来说,包括CSI、频偏、RSSI、调制方式。
	在第三章需求分析中我们提到,CSI和RSSI是安全研究中最常用的的物理层信息,频偏在基于身份验证的安全机制中也常常被用到。
	调制方式是指物理层数据处理过程中的信源调制方式,对于802.11协议,调制方式有BPSK、QPSK等,
	调制方式是物理层信息的重要参考,提取方式较为简单,分析物理层帧结构,从信令字段(SIGNAL)可以直接得到。
	而RSSI可以通过FMCOMMS3子板的嵌入式驱动程序得到,我们不需要进行提取,直接提供编程接口。
	以下将分别对CSI和频偏的提取过程进行介绍。

		\begin{figure}
			\centering
			\includegraphics[width=0.8\textwidth]{img/GRT_phy_rx.png}
			\caption{GRTSEC物理层接收端模块示意图}
			\label{fig:GRT_phy_rx}
		\end{figure}

	首先介绍CSI。作为OFDM通信系统,802.11的CSI是指各个子载波当前的信道状况。
	我们假设在一帧的接收过程之内,CSI大致不变,802.11协议也是基于此假设使用训练字段对信道进行估计。
	GRTSEC的物理层接收端模块示意图见\ref{fig:GRT_phy_rx},我们在信道估计模块提取到CSI。
	图\ref{fig:80211_spec_preamble}介绍了802.11a/g物理层训练字段的结构,
	训练字段是发送端发送一段双方已知的序列,接收端根据接收序列来做同步、信道估计等。
	802.11a/g物理层帧结构规定了两个训练字段,先是160个采样点的短训练字,后是160个采样点的长训练字,
	在长训练字后面是80个采样点的SIGNAL字段,里面规定了帧长度和调制方式。
	短训练字以16个采样点为周期,共10个周期,采样点间隔为0.05$us$,用来做AGC(自动增益控制)、时间同步、频率同步等,
	在后面介绍提取频偏和RSSI时会进一步说明。
	长训练字前32个采样点是保护间隔,抽取$T_1$、$T_2$各16个采样点组成的,
	$T_1$、$T_2$是重复的。长训练字用来做信道估计、细粒度的频偏估计。
		\begin{figure}
			\centering
			\includegraphics[width=1.0\textwidth]{img/80211_spec_preamble.png}
			\caption{802.11前导码结构}
			\label{fig:80211_spec_preamble}
			\cite{ieee80211}
		\end{figure}

	我们从长训练字中提取CSI。长训练字的采样点为1与-1的序列,如式\ref{fig:80211_lts_sequence},不包含其他值(中间的0为直流分量,不承载信息),
	因此接收值只需转换符号即可得到各个子载波的信道状况。而短训练字中除了1与-1外,还有0和1+j这样的值,不适合提取CSI。
	具体实现时,我们预存长训练字的理论值,与接收到的长训练字进行比较,理论值为-1时对接收值转换符号,理论值为1时不变。
	符号转换后得到52个采样值组成的向量,这个向量便可以代表此帧的CSI。

		\begin{figure}
			\centering
			\includegraphics[width=0.9\textwidth]{img/80211_lts_sequence.png}
			\caption{802.11前导码长训练字采样值}
			\label{fig:80211_lts_sequence}
			\cite{ieee80211}
		\end{figure}

	然后介绍频偏的提取。频偏在频偏估计模块中提取,利用短训练字的周期性,具体来说,
	使用短训练字的后64个采样点,即10个周期中的后4个周期,原因有两点,
	一是前面的采样点会受AGC(Automatic Gain Control,自动增益控制)的影响,幅度有波动,
	二是大多数时候从短训练字的中间位置同步到一帧,前面部门有可能同步不到,而后4个周期实践表明是切实可行的。
	我们首先对频偏公式进行变换,此处的推导比第二章更加详细,并结合具体实现。
	假设$k_1$、$k_2$时刻的接收值为$r(k_1)$和$r(k_2)$,理论值为$s(k_1)$和$s(k_2)$,有以下公式:
		\begin{equation}
			\centering
			r(k_1)=s(k_1) \cdot e^{j2 \pi \Delta fk_1 / f_s}
			\label{fig:freq_offseet_rs_1}
		\end{equation}
		\begin{equation}
			\centering
			r(k_2)=s(k_2) \cdot e^{j2 \pi \Delta fk_2 / f_s}
			\label{fig:freq_offseet_rs_2}
		\end{equation}

	其中,$\Delta f$为要求的频偏,$f_s$为采样频率,在802.11a/g中数值为20M,在802.11n中有20M和40M两种情况。
	$r(k_1)$、$r(k_2)$、$s(k_1)$、$s(k_2)$、$f_s$均已知,我们可以通过以下方式求出。
	根据短训练字每16个采样点重复一次的特性可知,当$k_2 = k_1 + 16$时,$s(k_1) = s(k_2)$,
	代入前面的公式,可以推出以下两式:
		\begin{equation}
			\centering
			def: \frac{r(k_1)}{r(k_2)} = A + Bj
			\label{fig:freq_offseet_AB_def}
		\end{equation}
		\begin{equation}
			\centering
			\Delta f = \frac{f_s}{2\pi \cdot 16} \cdot arctan(\frac{B}{A})
			\label{fig:freq_offseet_result}
		\end{equation}

	式\ref{fig:freq_offseet_AB_def}由复数的性质推出,$r(k_1)$、$r(k_2)$都是复数,相除的结果也是复数,
	复数总是可以写成$A + Bj$的形式,其中$A$、$B$为实数。
	式\ref{fig:freq_offseet_result}是将式\ref{fig:freq_offseet_AB_def}代入
	式\ref{fig:freq_offseet_rs_1}、\ref{fig:freq_offseet_rs_2}得出,得到频偏值。

	在具体实现时,我们在硬件中得到$A$、$B$,作为物理层信息传给软件,再由软件计算$arctan$和常数乘法得到频偏。
	硬件计算$A$、$B$时,需要对每隔16个的接收采样点进行相除,
	复数相除$\frac{r(k_1)}{r(k_2)}$可以转化为$\frac{1}{|r(k_2)|^2}r(k_1) \cdot r(k_2)^* $。
	每一个采样点都可以跟它之前的第16个采样点得到一组$A$、$B$,我们对64个采样点得到的64个$A$、$B$进行相加求平均,降低误差,
	最终提取到的物理层信息即为平均的$A$、$B$,分别代表周期性偏移实部和周期性偏移虚部。

	\section{物理层信息软件编程接口}\label{sec:grtsec_phyinfo_api}
	此小节介绍如何将物理层信息从硬件模块提取到软件中。
	GRTSEC的软件代码分为嵌入式软件和上位主机的驱动程序,软件编程接口面临的问题是,
	有研究需要在物理层或LOW MAC层使用这些物理层信息,而另有研究只在主机上进行操作,
	如何满足这两种需求。

	我们采用的方法是嵌入式软件与主机软件的两级编程结构,
	将物理层信息首先提取到嵌入式软件中,提供嵌入式软件的编程接口,然后通过USB通信库将物理层信息传递给主机,
	而不是将物理层信息直接引出到主机驱动程序中。这样做有以下几点好处:
		\begin{itemize}
			\item LOW MAC的控制流程由嵌入式软件实现,用户有可能在LOW MAC中用到物理层信息,
			比如根据物理层信息决定是否回复ACK;
			\item 嵌入式软件包含了与主机驱动程序交互的接口,物理层信息先提取到嵌入式软件中,
			用户决定是否继续将信息继续上传,保持控制流程的集中性;
			\item 硬件逻辑与嵌入式软件的交互方便扩展,倘若用户希望引出其它信息,增加AXI寄存器即可,
			但如果直接引出到驱动程序,还需要对USB通信库的硬件逻辑进行修改,扩展不便,理论上不应对通信库进行过多修改。
		\end{itemize}

	下面介绍具体实现。
	GRTSEC的软硬件之间通过AXI寄存器传递数值,每个寄存器为32位,AXI寄存器定义见表\ref{tab:grtsec_axi_reg_define}。
	与同步相关的寄存器的含义见\ref{sec:grtsec_phyinfo_extension}小节。
	可以看到为了从32位的AXI寄存器传递到软件,我们提供的是原始数据的接口,
	用户得到原始数据后可以选择在嵌入式软件内计算得到最终数据,我们提供了从原始数据得到最终数据的Python示例代码,
	比如由周期性偏移的实部和虚部得到频偏值,
	也可以选择根据原始数据进行处理,很多情况下原始数据比计算结果更有意义。

	\begin{table}[!hbp]
	\centering
	\caption{GRTSEC软件编程接口的寄存器定义}
	\label{tab:grtsec_axi_reg_define}
		\begin{tabular}{|l|l|l|} \hline
		寄存器号 & 读写方向 & 寄存器定义 \\ \hline
		0x00 & W & 请求读取物理层信息 \\ \hline
		0x01 & R & 物理层信息已准备好 \\ \hline
		0x02 & R & 周期性偏移实部,高32位 \\ \hline
		0x03 & R & 周期性偏移实部,低32位 \\ \hline
		0x04 & R & 周期性偏移虚部,高32位 \\ \hline
		0x05 & R & 周期性偏移虚部,高32位 \\ \hline
		0x06 & R & 同步自相关性求和分子 \\ \hline
		0x07 & R & 同步自相关性归一化系数 \\ \hline
		0x08 & R & 同步与短训练字互相关性求和分子 \\ \hline
		0x09 & R & 同步与长训练字互相关性求和分子 \\ \hline
		0x0A & R & 同步互相关性归一化系数 \\ \hline
		0x0B & R & 信令字段,包含帧长度和调制方式 \\ \hline
		0x0C & R & CSI,-21号子载波 \\ \hline
		0x0D & R & CSI,-7号子载波 \\ \hline
		0x0E & R & CSI,+7号子载波 \\ \hline
		0x0F & R & CSI,21号子载波 \\ \hline
		0x10 & R & 时间戳,高4字节 \\ \hline
		0x11 & R & 时间戳,低4字节 \\ \hline
		\end{tabular}
	\end{table}

	\section{物理层信息硬件分析模块}\label{sec:grtsec_phyinfo_analyzer}
	如果物理层安全研究者希望将新安全机制部署在实际通信通信系统中,仅靠软件分析是不够的,
	在很多场景下,用软件进行数据处理的效率低,无法满足实时性的要求。
	比如加密协议的研究者,加密算法往往复杂度高,用软件计算会成为系统性能的瓶颈。
	因此GRTSEC提供了一个硬件数据分析模块,用于对物理层信息进行数据分析,

	分析模块的位置见\ref{fig:grtsec_hw_design},与物理层相连,得到物理层信息并进行数据处理,
	嵌入式软件通过AXI接口对数据处理过程进行参数配置,以及得到处理后的数据。
	硬件分析模块的好处是效率高,实时性强,用户认为计算密集或要求实时性的计算可以由硬件分析模块实现。

	在硬件分析模块,我们提供了32组AXI寄存器,用户可以进行软件与硬件的交互。
	例如,由嵌入式软件从物理层信息中提取出密钥,软件通过AXI寄存器将密钥传给硬件分析模块,使用硬件分析模块实现解密算法。
	硬件高效且可自定义的特点非常适合实现此类算法。

	\section{物理层信息的扩展方法}\label{sec:grtsec_phyinfo_extension}
	提供了常用物理层信息的软硬件编程接口后,我们面临另一个问题,
	研究总是不断创新与突破的,验证平台不能只满足已有研究,还要提供一定的扩展性。

	因此,GRTSEC在提供WiFi安全研究常用的物理层信息如CSI、频偏、RSSI、调制方式之外,也提供了用户进行扩展的方法。
	加上GRT2.0本身具有的物理层全可编程的特点,GRTSEC可以很好地适用研究者日益演进且多变的需求。

	下面结合实例阐述物理层信息的扩展方法。
	MobiCom08\cite{mobicom08radiometric}中除了用到频偏外,还使用同步相关性(SYNC correlation)
	以及星座点偏移(I/Q origin offset)来生成设备指纹。
	我们接下来以同步相关性为例,介绍用户如何进行物理层信息的扩展。

	首先是同步相关性的背景。我们用短训练字进行时间同步,利用短训练字每16个采样点重复一次的周期性,
	当相临16个采样点之间的相关性(自相关性)超过一定阈值时,我们认为发现了短训练字序列,称作帧同步。
	帧同步的过程如图\ref{fig:sync_sliding_window},我们采用两个相邻的滑动窗口$W_1$和$W_2$,窗口大小为16个采样点。
	但只同步到帧是不够的,有10组重复的短训练字,我们无法得知同步到了哪相邻两组,需要确切定位到采样点。
	我们利用了长训练字相关性,把16个采样点与长训练字进行比较,与长训练字的相关性(互相关性)超过一定阈值,且自相关性低于阈值时,
	认为发现了短训练字的结束、长训练字的开始,可以精确到采样点,称作符号同步。
	符号同步的过程如图\ref{fig:sync_sliding_window_lts},当$W_1$内的采样点全部位于短训练字部分,$W_2$内的采样点全部位于长训练字部分时,
	$W_2$内采样点与长训练字前16个点的理论值的互相关性会高于一定阈值,且$W_1$与$W_2$之间点的相关性会低于阈值。
	同步相关性包括自相关性和与互相关性,
	MobiCom08\cite{mobicom08radiometric}中指出,同步的互相关性与硬件有关,可以用来做设备的识别。

		\begin{figure}
			\centering
			\includegraphics[width=0.4\textwidth]{img/sync_sliding_window.png}
			\caption{帧同步时滑动窗口示意图}
			\label{fig:sync_sliding_window}
		\end{figure}

		\begin{figure}
			\centering
			\includegraphics[width=0.4\textwidth]{img/sync_sliding_window_lts.png}
			\caption{符号同步时滑动窗口示意图}
			\label{fig:sync_sliding_window_lts}
		\end{figure}

	下面介绍如何添加同步相关性的软件接口,以互相关性为例介绍,自相关性与之类似。

	第一步,列出同步互相关性的公式,分析哪些计算适合由硬件完成,哪些计算适合引出到软件由软件完成。
	\ref{math:sync_inter_correlation}是互相关性的公式,其中$P_n$是归一化系数。
		\begin{equation}
			\centering
			C_n = \sum_{k=0}^{L-1} r_{n+k} \cdot s_{n+k}^*
		\end{equation}
		\begin{equation}
			\centering
			P_n = \sum_{k=0}^{L-1} |r_{n+k}|^2 \cdot \sum_{k=0}^{L-1} |s_{n+k}|^2
		\end{equation}
		\begin{equation}
			\centering
			M_n = \frac{|C_n|^2}{P_n}
			\label{math:sync_inter_correlation}
		\end{equation}

	我们知道FPGA硬件适合做乘法与加法,不适合做除法,除法需要多周期,延迟高。因此,我们可以把计算$C_n$与$P_n$交由硬件完成。
	另外,因为长训练字的能量值$\sum_{k=0}^{L-1} |s_{n+k}|^2$是固定的,我们只需在硬件中计算$\sum_{k=0}^{L-1} |r_{n+k}|^2$,称作$P'_n$,
	计算$M_n$的过程由软件完成。

	第二步,在同步模块内计算得到$C_n$、$P'_n$,并从同步模块引出。
	添加同步模块的端口,如下所示,这里只是示意代码,实际上除了$C_n$、$P'_n$外,我们还引出了计算自相关性的几个信号。
	\begin{lstlisting}[language={Verilog}]
module rx_synchronization(
... // other ports
input signal_valid,
output [31:0] sync_para_C,
output [31:0] sync_para_P1,
... // other ports
);
	\end{lstlisting}
	signal\_valid信号是指后续模块解出了signal字段,只有signal字段符合要求时此值为1,否则为0。
	我们在signal\_valid为1时为sync\_para\_C、sync\_para\_P1赋值。

	第三步,做时钟域转换,把同步模块对应时钟域下的sync\_para\_C、sync\_para\_P1,
	转化为AXI总线时钟域下的axi\_sync\_para\_C、axi\_sync\_para\_P1,并赋值给AXI寄存器。
	GRTSEC提供了一个时钟域转换模块reg\_clk\_domain\_switch,使用方法如下所示,我们将$C_n$、$P'_n$合并后一起转时钟。
	\begin{lstlisting}[language={Verilog}]
reg_clk_domain_switch_64 SEC_SYNC_clk_switch_clk0_to_clkaxi(
.clk_pre(usr_clk_ch0),
.clk_pos(S_AXI_ACLK),
.rst_pre(usr_rst_ch0),
.rst_pos(~S_AXI_ARESETN),
.reg_pre(sync_para),
.reg_pos(axi_sync_para)
);
	\end{lstlisting}

	第四步,从嵌入式软件代码读取$C_n$、$P'_n$,并用软件计算出我们所需的$M_n$。
	嵌入式软件读取和计算代码如下所示,SYNC\_PARA\_LTS为预存好的长训练字的能量值。
	\begin{lstlisting}[language={C}]
int syncParaC = Xil_In32(PHYSyncPara1);
int syncParaP1 = Xil_In32(PHYSyncPara2);
float syncParaM = syncParaC * syncParaC / (syncParaP1 * SYNC_PARA_LTS);
	\end{lstlisting}

	至此,以同步相关性为例,用户完成了物理层信息的扩展。

	\section{多射频模式}\label{sec:grtsec_loopback}
	在实际的WiFi安全使用样例设计中,我们发现一个问题,在真正通过空气进行通信之前,研究者往往会先验证数据处理流程是否能在硬件上正常工作,
	类似于将射频前端的发送端与接收端通过馈线相连,因此需要验证平台提供对完美信道进行模拟的能力,但原有的GRT2.0的射频通信库不支持这一点。
	完美信道是指将发送端和接收端之间连接在一起,接收到的信号与发送信号完全相同,
	用以验证物理层信息是否是完美信息,比如CSI全部为1,频偏为0,同步相关性为1等,
	以及模拟在完美通信时新数据处理算法是否可行,排除信号偏差带来的影响,
	比如验证加密解密算法在硬件上运行是否正确,不需要真正地与射频前端连接起来。

	为了支持完美信道,我们对射频通信库进行了扩展,提供两种模式,射频模式和回环模式。
	射频模式是原射频通信库采用的方式,与射频前端相连。回环模式是将发送数据直接反馈给接收端。

	这样做的好处是,WiFi安全的研究者在研究的前期系统实现与调试阶段,使用回环模式对完美信道进行模拟,
	不需要购买射频前端,不需要两台设备互相通信,便可以得到一些测试结果,对系统实现进行调试。
	搭配射频前端后,只需要修改一个参数,就能迅速适配射频前端,在空气中完成通信实验。

		\begin{figure}
			\centering
			\includegraphics[width=0.7\textwidth]{img/rflib_loopback_mux.png}
			\caption{射频通信库射频模式与回环模式示意图}
			\label{fig:rflib_loopback_mux}
		\end{figure}

	下面介绍回环模式的具体实现,示意见图\ref{fig:rflib_loopback_mux}。
	射频通信库与物理层的数据接口是两个异步FIFO,在回环模式中,我们将发送端FIFO的输出,直接接到接收端FIFO的输入上。
	当有数据来时,向接收FIFO中填充空白信号,模拟空气中的无信号时的空白时间,当物理层发送端告知射频通信库要发送一帧时,
	进入回环模式,从发送端FIFO取数据,直接放到接收端FIFO中。

	值得注意的是,除了数据流需要做多路选择,时钟和复位信号也要做多路选择。
	收发FIFO都是异步FIFO,在射频模式时,射频端的时钟和复位信号来自射频模块,在回环模式时,需要改成物理层时钟,
	如下所示。
	\begin{lstlisting}[language={Verilog}]
assign clk = (MODE == RFD_MODE_LOOPBACK)? PHY_clk : rf_clk;
assign rst = (MODE == RFD_MODE_LOOPBACK)? PHY_rst : rf_rst;
	\end{lstlisting}

	\section{本章小结}\label{sec:grt2.0_summary}
	本节介绍了GRTSEC在GRT2.0系统上的改进,
	首先在\ref{sec:grtsec_overview}小节介绍了总体设计,
	面对GRT2.0嵌入式软件与物理层硬件交互不便,研究者无法通过软件获取物理层信息的问题,
	我们提出将物理层通过AXI总线与嵌入式处理器相连的解决方案,将物理层信息从硬件中提取出来,
	提供软件编程接口,并设计了一个硬件分析模块对安全研究进行支持,
	这样做的好处是方便WiFi安全的研究者获取常见的物理层信息,可以扩展其它物理层信息,可以使用硬件分析模块对算法进行加速。

	\ref{sec:grtsec_phyinfo_substract}小节介绍了CSI、频偏、RSSI、调制方式的提取方法,
	面对帧对齐和精确接收问题,我们采用物理层信息的三级缓存方法和添加时间戳的方法。
	\ref{sec:grtsec_phyinfo_api}小节介绍了物理层信息的软件接口,
	设计了嵌入式软件和主机软件的两级编程结构,满足了研究者在不同层编程的需求。
	\ref{sec:grtsec_phyinfo_analyzer}小节介绍了硬件分析模块的设计,
	提供了一个硬件分析模块对软件处理过程进行加速,适用于加密解密等计算密集的安全算法。
	\ref{sec:grtsec_phyinfo_extension}小节以同步相关性为例介绍了用户如何扩展物理层信息。
	\ref{sec:grtsec_loopback}小节介绍了为提高开发调试的效率,射频通信库模块增加回环模式,提供对完美信道的支持。

	% Copyright (c) 2008-2009 solvethis
% Copyright (c) 2010-2011 Casper Ti. Vector
% Public domain.

\chapter{性能测试与评估}\label{chap:envaluation}
本章将对支持物理层WiFi安全研究的验证平台GRTSEC进行评估。
首先在\ref{sec:envaluation_env_setup}节介绍测试环境,介绍开发板型号、开发套件版本号等,
其次在\ref{sec:envaluation_performance}节对GRTSEC的性能进行测试,
然后在\ref{sec:envaluation_demo}节提出多个WiFi安全使用样例,并对GRTSEC在使用样例中的表现进行评测,
最后在\ref{sec:envaluation_summary}节论证了GRTSEC可以满足WiFi安全研究的需求,并且对本章进行总结。

  \section{测试环境}\label{sec:envaluation_env_setup}
  在第四章我们介绍了GRTSEC的前期工作GRT2.0的组成部分,GRTSEC与GRT2.0的组成部分相同,但具体测试环境略有不同。
  在GRTSEC中,我们使用的上位主机是一台支持USB3.0接口、运行有Ubuntu14.04操作系统的计算机,
  FPGA是一块Xilinx VC707板卡\cite{xilinxvc707},
  配置计算机是任意一台支持Xilinx Vivado开发套件的计算机,Windows系统或Linux系统均可,
  射频前端是Analog Device公司的EVAL-AD-FMCOMMS3-EBZ开发板\cite{fmcomms3}。
  FPGA开发工具我们使用的是Xilinx Vivado 2015.2,FPGA嵌入式软件开发工具是Xilinx SDK 2015.2。

  \section{平台性能测试}\label{sec:envaluation_performance}
  本节对GRTSEC平台的性能进行测试,分成两个小节,
  \ref{subsec:envaluation_comparation}小节对GRTSEC与GRT2.0在资源使用率和性能上进行比较,
  \ref{subsec:envaluation_phy_info}小节对GRTSEC在获取物理层信息上的有效性进行测试。

    \subsection{GRTSEC与GRT的对比测试}\label{subsec:envaluation_comparation}
    在本小节我们对GRTSEC与GRT2.0在资源使用率和性能上进行比较。
    GRTSEC相对于GRT2.0在代码上的改动主要有以下几方面:
    	\begin{itemize}
        \item 在物理层中添加AXI接口,打包为IP核,与处理器相连;
    		\item 修改物理层中的同步、频偏、信道估计模块,修改顶层模块,添加物理层信息的三级缓存逻辑及其时钟域转换逻辑;
        \item 修改射频通信库模块,增加回环模式;
    		\item 添加硬件分析模块IP,通过AXI总线与处理器相连;
        \item 修改嵌入式软件代码,增加物理层信息的接口;
        \item 对于使用样例,添加对物理层信息进行数据分析的Python代码。
    	\end{itemize}

    与GRT2.0相比,GRTSEC的代码变化如表\ref{tab:envaluate_code_comparation}所示。
    GRT2.0实现了完整的802.11协议,可以看到,基于GRT2.0进行开发可以付出相对较小的工作量达到目标,体现了GRT2.0的可扩展性。
      \begin{table}[!hbp]
      \centering
      \caption{GRTSEC与GRT2.0的代码比较}
      \label{tab:envaluate_code_comparation}
        \begin{tabular}{|l|p{0.2\textwidth}|p{0.2\textwidth}|l|} \hline
        代码 & GRT2.0的代码行数 & GRTSEC修改或增加代码行数 & 变化比例(\%)\\ \hline
        物理层硬件Verilog代码 & 34233 & 987 & 2.9 \\ \hline
        射频通信库硬件Verilog代码 & 175 & 20 & 11.4 \\ \hline
        硬件分析模块硬件Verilog代码 & / & 130 & / \\ \hline
        嵌入式软件C代码 & 2573 & 44 & 1.7 \\ \hline
        使用样例Python代码 & / & 274 & / \\ \hline
        \end{tabular}
      \end{table}

    下面展示的是FPGA的资源占用率,GRT2.0的资源占有率见表\ref{tab:envaluate_grt2.0_resource},
    增加GRTSEC的扩展后,资源占用率见表\ref{tab:envaluate_grtsec_resource}。
      \begin{table}[!hbp]
      \centering
      \caption{GRT2.0在VC707开发板上的资源占用率}
      \label{tab:envaluate_grt2.0_resource}
        \begin{tabular}{|l|l|l|l|} \hline
        资源 & 已使用 & 总数 & 占用率 \\ \hline
        Flip-Flop & 76781 & 607200 & 12.65 \\ \hline
        LUT & 98657 & 303600 & 32.50 \\ \hline
        Memory LUT & 4551 & 130800 & 3.48 \\ \hline
        I/O & 224 & 700 & 32.00 \\ \hline
        BRAM & 319 & 1030 & 30.97 \\ \hline
        DSP48 & 294 & 2800 & 10.50 \\ \hline
        BUFG & 17 & 32 & 53.12 \\ \hline
        \end{tabular}
      \end{table}

      \begin{table}[!hbp]
      \centering
      \caption{GRTSEC在VC707开发板上的资源占用率}
      \label{tab:envaluate_grtsec_resource}
        \begin{tabular}{|l|l|l|l|} \hline
        资源 & 已使用 & 总数 & 占用率 \% \\ \hline
        Flip-Flop & 79564 & 607200 & 13.10 \\ \hline
        LUT & 95761 & 303600 & 31.54 \\ \hline
        Memory LUT & 4058 & 130800 & 3.10 \\ \hline
        I/O & 224 & 700 & 32.00 \\ \hline
        BRAM & 263 & 1030 & 25.53 \\ \hline
        DSP48 & 303 & 2800 & 10.82 \\ \hline
        BUFG & 14 & 32 & 43.75 \\ \hline
        \end{tabular}
      \end{table}

    Flip-Flop(触发器)和LUT(look-up table,查找表)是Xilinx 7系列FPGA的主要逻辑资源,
    Flip-Flop可用来做寄存器,LUT可用来做逻辑和存储,专门做存储的那部分称作Memory LUT。
    BRAM(Block RAM)是Xilinx 7系列FPGA的主要存储资源,多数FIFO是利用BRAM实现的,
    少数利用DRAM(Distribute RAM)实现,DRAM实际上消耗Memory LUT资源。

    我们可以看到,GRTSEC相对于GRT2.0,Flip-Flop资源使用有所增加但没有增加太多,其他资源如LUT和BRAM资源使用实际上还减少了,
    原因有以下几点:
    	\begin{itemize}
    		\item GRTSEC采用多级寄存器缓存技术进行帧对齐,因此会消耗更多的Flip-Flop,
        但相对于原GRT2.0的消耗只是很小的一部分,所以Flip-Flop使用率增加不多;
    		\item LUT有少量下降,因为原有GRT2.0系统设计中射频部分和LOW MAC都在block design中,
        而中间的物理层在block design之外,GRTSEC将物理层由block design外部移到了block design内部,并封装为一个IP核,
        有效地减少了block design内外部的连线所消耗的LUT资源;
    		\item GRTSEC对系统使用的FIFO进行了优化,减小一些FIFO不必要的深度,因此BRAM和Memory LUT使用率有所下降;
        \item BUFG是用来做时钟分频和去毛刺的,BUFG资源有所下降是因为GRTSEC对全局的时钟进行了优化,
        多处生成同一频率时钟的BUFG合并到一处生成,即相同频率的两个异步时钟合并为一个。
    	\end{itemize}

    然后对嵌入式软件代码的性能进行测试,主要体现在传输延迟上。从收到一帧到回复ACK的时间间隔,是最关键的延迟指标,
    如果延迟太高,会大大影响传输性能,留给用户可编程的空间会降低。
    GRTSEC相对GRT2.0增加了读取物理层信息的过程,本节对GRTSEC在这方面增加的负载进行了测量,
    测量方法是记录不读取物理层信息时软件代码执行时间,即GRT2.0的软件代码执行时间,再记录读取物理层信息时的软件代码执行时间,
    即GRTSEC的软件代码执行时间,多帧测量求平均后再做差。
    此处的软件代码代码执行时间是嵌入式软件处理一帧的总时间,包括了等待硬件计算CRC校验的时间,接收帧越长此时间越长,
    差值即为读取物理层信息所花的时间。
    性能结果见表\ref{tab:envaluate_grtsec_latency}。
    \begin{table}[!hbp]
    \centering
    \caption{GRTSEC与GRT2.0的延迟性能对比表}
    \label{tab:envaluate_grtsec_latency}
      \begin{tabular}{|l|l|l|} \hline
      测量对象 & 接收帧数 & 平均软件处理时间(us) \\ \hline
      GRT2.0软件接收端 & 3219 & 41.801802 \\ \hline
      GRTSEC软件接收端 & 3641 & 41.867069 \\ \hline
      读取物理层信息 & / & 0.065268 \\ \hline
      \end{tabular}
    \end{table}

    我们可以看到,延迟增加约65ns,增加的并不多,因为经过精心设计的硬件逻辑,可以保证在软件读取之前将物理层信息准备好,
    因此多花的时间即为读取多组AXI寄存器的时间,硬件开销很小。

    \subsection{GRTSEC获取物理层信息性能测试}\label{subsec:envaluation_phy_info}
    本小节对GRTSEC读取物理层信息的有效性进行测试,选用真实的802.11设备进行测试,地点是北京大学理科五号楼五楼。
    我们在2.4GHz频段的最常用1信道(2.412GHz)以及5GHz频段的161信道(5.805GHz),对该频段进行1-2分钟的监听,
    记录频偏、CSI、RSSI、调制方式等物理层信息,绘制成以下图表。
    这些图表的目的是展示GRTSEC读取物理层信息的有效性,WiFi安全的研究者可以物理层信息进行进一步地分析和处理,本文不做更多的分析。

      \begin{table}[!hbp]
      \centering
      \caption{1信道设备分布及频偏值表}
      \label{tab:envaluate_2412_frame_number}
        \begin{tabular}{|l|l|l|l|l|} \hline
        序号 & 设备MAC地址前12位 & 接收帧数 & 频偏平均值(rad) & 频偏标准差(rad) \\ \hline
        设备1  & 0x003A98 & 4247 & -26.5586 & 6.8343 \\ \hline
        设备2  & 0xA0C589 & 686  & 5.0964 & 6.7770 \\ \hline
        设备3  & 0xA434D9 & 556  & -24.5785 & 7.6319 \\ \hline
        设备4  & 0x66C002 & 507  & -0.1805 & 4.3990 \\ \hline
        设备5  & 0x48437C & 409  & -11.0173 & 5.1095 \\ \hline
        设备6  & 0x08D40C & 394  & 0.3264 & 3.7925 \\ \hline
        设备7  & 0x2ADCB6 & 364  & -1.1527 & 11.1968 \\ \hline
        设备8  & 0xDE436A & 229  & -27.3364 & 3.3673 \\ \hline
        设备9  & 0x8AF314 & 195  & -24.1565 & 6.2982 \\ \hline
        设备10 & 0xF48B32 & 151  & 6.2376 & 7.9951 \\ \hline
        其它设备 & /      & 1218 & -15.9206 & 13.1757 \\ \hline
        \end{tabular}
      \end{table}

    表\ref{tab:envaluate_2412_frame_number}为1信道下接收帧的设备分布及其频偏值,
    可以看到,不同设备之间频偏相差较大,除设备7外,同一设备的频偏标准差小,其它一行的频偏标准差大,进一步说明不同设备之间频偏相差较大。

      \begin{table}[!hbp]
      \centering
      \caption{161信道设备分布及频偏值表}
      \label{tab:envaluate_5805_frame_number}
        \begin{tabular}{|l|l|l|l|l|} \hline
        序号 & 设备MAC地址前12位 & 接收帧数 & 频偏平均值(rad) & 频偏标准差(rad) \\ \hline
        设备A  & 0xDCEF09 & 4953 & -4.8232 & 9.7126 \\ \hline
        设备B  & 0x9DA868 &  711 & -3.4210 & 4.7535 \\ \hline
        设备C  & 0x67A622 &  672 & -4.0277 & 7.7692 \\ \hline
        设备D  & 0x9AE07C &  291 & -3.7384 & 8.9364 \\ \hline
        设备E  & 0xD6FF00 &  189 & -3.9803 & 5.4527 \\ \hline
        设备F  & 0xB699D3 &   83 & -2.8351 & 1.9508 \\ \hline
        设备G  & 0x7ADC74 &   83 & -62.3400 & 24.0868 \\ \hline
        设备H  & 0x2C9FC7 &   56 & -5.5038 & 13.2189 \\ \hline
        其它设备 & /      &  575 & -18.3011 & 37.5738 \\ \hline
        \end{tabular}
      \end{table}

    表\ref{tab:envaluate_5805_frame_number}为161信道下接收帧的设备分布及其频偏值,
    在5GHz频段下,设备7与其它设备之间频偏相差很大,
    相比2.4GHz,其它一行的频偏标准差非常大,达到了37,进一步说明不同设备之间频偏相差较大。

    图\ref{fig:envaluation_csi_2_4g}与图\ref{fig:envaluation_csi_5g}分别展示了在2.4GHz和5GHz下两台设备的CSI随时间的变化,
    两台设备分别是前面提到的两个信道下出现最多的设备,接收帧数分别为4247帧和4953帧,监听时间都在10秒以上,
    选取了CSI中两个子载波-21与+21的幅度响应进行展示,这两个子载波也是802.11协议中规定的导频子载波。
    由两个图可以看到,除去因环境变化引起骤降和骤增,总体幅度响应的趋势是缓慢变化的。
  		\begin{figure}[H]
  			\centering
  			\includegraphics[width=1.0\textwidth]{img/envaluation_csi_2_4g.png}
  			\caption{2.4GHz下某设备子载波幅度响应曲线}
  			\label{fig:envaluation_csi_2_4g}
  		\end{figure}

  		\begin{figure}[H]
  			\centering
  			\includegraphics[width=1.0\textwidth]{img/envaluation_csi_5g.png}
  			\caption{5GHz下某设备子载波幅度响应曲线}
  			\label{fig:envaluation_csi_5g}
  		\end{figure}


    图\ref{fig:envaluation_rssi_2_4g}与图\ref{fig:envaluation_rssi_5g}分别展示了在2.4GHz和5GHz下各两台设备的RSSI随时间的变化,
    记录的是前面提到的两个信道下出现最多的两台设备,2.4GHz下是设备1和设备2,5GHz下是设备A和设备B。
    可以看到2.4GHz下的两台设备的RSSI差别较大,5GHz下的这两台设备RSSI差别较小。
  		\begin{figure}[H]
  			\centering
  			\includegraphics[width=1.0\textwidth]{img/envaluation_rssi_2_4g.png}
  			\caption{2.4GHz下两台设备RSSI随时间变化曲线}
  			\label{fig:envaluation_rssi_2_4g}
  		\end{figure}

  		\begin{figure}[H]
  			\centering
  			\includegraphics[width=1.0\textwidth]{img/envaluation_rssi_5g.png}
  			\caption{5GHz下两台设备RSSI随时间变化曲线}
  			\label{fig:envaluation_rssi_5g}
  		\end{figure}

    以上呈现的只是一部分测试数据的简单分析结果,目的是展示GRTSEC对物理层信息的提取能力,
    实际上在第\ref{chap:background}章提到,物理层信息在WiFi安全的研究中有广泛的应用,待GRTSEC平台的使用者去发掘。

  \section{样例测试}\label{sec:envaluation_demo}
	本节会通过两个实际的物理层WiFi安全研究的样例,进一步说明GRTSEC平台的使用场景和方法,
	同时,用户也可以基于这些使用样例开展自己的研究。
	在\ref{subsec:demo_fake_wifi}小节会介绍如何使用GRTSEC搭建伪装WiFi,并用伪装WiFi发起钓鱼攻击,获取用户的敏感信息,
	在\ref{subsec:demo_phy_auth}小节会提出一种利用利用硬件指纹技术对不同WiFi设备进行识别的方法。

    \subsection{搭建伪装AP}\label{subsec:demo_fake_wifi}

    身份伪装是很多WiFi攻击的前提和基础,比如DoS攻击、会话劫持、中间人攻击、数据篡改、窃听等\cite{ieeewc10noncryp}。
    针对身份伪装的安全研究需要先搭建伪装AP,本节设计了搭建伪装AP的样例,并用伪装AP发起钓鱼攻击,获取用户的敏感信息。
    下面将通过一个实际的例子说明伪装AP的用途。

		在北京大学,计算中心为全校师生员工提供了无线校园WiFi,名字是Wireless PKU,遍布教学楼、实验室、食堂等场所。
		Wireless PKU是无密码的开放WiFi,用户连接到Wireless PKU后,首先跳转到一个认证网页,输入网关的账号、密码进行登录,
		登录后才可以继续上网。对于学生来说,网关账号、密码同时也是其他校园网页登录的账号、密码,
		包括信息登记、成绩查询、选课申请等都使用同一套认证系统。
		假如密码泄露会造成巨大的损失,轻则已选的课程被退掉或者成绩泄露,重则学籍信息被篡改,无法毕业。
		在本小节,我们将利用GRTSEC搭建AP,伪装成Wireless PKU,并制作一个钓鱼认证网页,用来盗取统一认证账号。

		真实Wireless PKU的AP与伪装AP的分布场景如图\ref{fig:demo_fake_wifi_scenary}所示,
		这张图真实地模拟了北京大学理科五号楼五楼的AP分布,其中AP1、AP2、AP3是北京大学计算中心布置的Wireless PKU的AP,
		GRTSEC摆放在房间1中。假设AP1、AP2、AP3的MAC地址是$MA_1$、$MA_2$、$MA_3$。
		房间1内有个用户,我们称他为小明,在GRTSEC发起伪装之前,小明可以看到信号较强的AP1和信号弱一些的AP2,看不到远处的AP3。
		为了让小明能够连接上GRTSEC,我们将GRTSEC的SSID设置为Wireless PKU,将MAC地址设置为$MA_3$,
		即使小明用MAC地址的白名单来分辨伪装AP也是无法识别的。小明连接上GRTSEC搭建的AP后,会先看到钓鱼认证网页,
		习惯了上网认证的小明会毫无戒心的输入自己的统一认证账号和密码。

		此样例旨在说明设计目标中的两点:可与商用802.11设备实时通信,在这里是与小明的手机相连;
		可与上层网络协议栈连接,在这里是与应用层的钓鱼认证网页相连。

			\begin{figure}
				\centering
				\includegraphics[width=0.7\textwidth]{img/demo_fake_wifi_scenary.png}
				\caption{伪装AP与真实AP分布场景图}
				\label{fig:demo_fake_wifi_scenary}
			\end{figure}

		具体实现时,首先,在上位主机上搭建钓鱼认证网页,钓鱼网页参考了本组张高瀚同学做的网页\cite{zgh17wifi},
		然后,修改主机的网络配置,使网络请求都转跳到钓鱼认证网页,
		最后,将GRTSEC运行在AP模式,等待进入房间1的用户连接。

    下面将通过实际测试结果说明其有效性。
    如图\ref{fig:envaluation_fake_ap}所示,我们利用GRTSEC搭建伪装AP,名字为Wireless PKU GRTSEC,
    可以被一部手机连接,手机自动可以跳转到钓鱼页面。
  		\begin{figure}[H]
  			\centering
  			\includegraphics[width=1.0\textwidth]{img/grtsec_fake_ap.png}
  			\caption{伪装AP与钓鱼认证网页}
  			\label{fig:envaluation_fake_ap}
  		\end{figure}

    使用WiFi信号分析工具WirelessMon\cite{wirelessmon}对周边WiFi进行查看,如图\ref{fig:envaluation_fake_ap_wirelessmon},
    可以看到Wireless PKU和Wireless PKU SEC同时存在,且Wireless PKU SEC的MAC地址伪装成了与AP3的MAC地址,与AP1、AP2的MAC地址很相近。
  		\begin{figure}[H]
  			\centering
  			\includegraphics[width=1.0\textwidth]{img/grtsec_wirelessmon.png}
  			\caption{使用WirelessMon查看伪装AP}
  			\label{fig:envaluation_fake_ap_wirelessmon}
  		\end{figure}

    我们可以看到,使用GRTSEC搭建的伪装AP足够以假乱真。
    用户手机真实连上了GRTSEC搭建的伪装AP,说明GRTSEC可与商用设备实时通信,
    用户手机看到了钓鱼认证网页,说明GRTSEC兼容上层网络应用。

    \subsection{利用物理层信息识别不同WiFi设备}\label{subsec:demo_phy_auth}
  	在802.11协议中,WiFi身份的标识是MAC地址,然而MAC地址非常容易伪装,利用笔记本电脑,
  	或者一部root过的安卓手机,搭配一些软件工具,就可以任意修改自己的MAC地址\cite{NetworkSecurity11hacking}。
  	% 在本节,我们提出一种利用物理层信息识别WiFi设备的方法,具体来说,使用物理层信息中的频偏和时钟偏移,识别不同的WiFi设备硬件,
  	% 弥补仅靠MAC地址进行识别的缺陷。

		下面通过一个实际的例子说明利用物理层信息识别不同WiFi设备的用途。
    一些用户家里的WiFi偶然被邻居知道了密码,又不想修改密码使自己的全部设备重新输入一次,我们假设这样的一个用户叫小明。
		为了防止邻居的“蹭网”,小明会对路由器进行设置,只允许特定MAC地址的设备连接,目前很多路由器支持这种模式。
		但是,不怀好意的邻居可以监听到小明设备的MAC地址,将自己的设备伪装成小明的设备骗过路由器,继续进行蹭网。
		小明希望做的是对自己的设备进行信任而不是对MAC地址进行信任,在本节,我们利用GRTSEC从接收到的包中提取频偏和时钟偏移,
		生成与WiFi设备对应的硬件指纹,这样小明可以在路由器端选择对特定硬件指纹进行信任,防止邻居继续蹭网。

		本样例主要关注对不同WiFi设备的识别和区分,并不针对无线路由器识别用户设备,实际上这种方法是通用的,
		可用于AP识别STA,可用于识别\ref{subsec:demo_fake_wifi}节提出的伪装AP,也可用于STA之间的互相识别。

      \begin{table}[!hbp]
      \centering
      \caption{不同的STA的频偏值}
      \label{tab:envaluate_freq_offset}
        \begin{tabular}{|l|l|l|l|} \hline
        设备 & 接收帧数 & 频偏平均值(rad) & 频偏标准差(rad) \\ \hline
        华为手机 & 451 & 14.8190 & 2.9401 \\ \hline
        iPhone & 223 & -29.0859 & 2.8856 \\ \hline
        小米手机 & 512 & 9.8436 & 2.3449 \\ \hline
        \end{tabular}
      \end{table}

      \begin{table}[!hbp]
      \centering
      \caption{利用机器学习方法识别不同的STA}
      \label{tab:envaluate_identify_sta}
        \begin{tabular}{|l|l|} \hline
        模型或算法 & 正确率(\%) \\ \hline
        k最近邻算法 & 98.15 \\ \hline
        线性回归分析 & 98.48 \\ \hline
        随机森林算法 & 97.31 \\ \hline
        决策树回归 & 96.80 \\ \hline
        支持向量机 & 97.98 \\ \hline
        \end{tabular}
      \end{table}

		下面介绍具体实现。
    首先,我们利用GRTSEC监听周边WiFi设备发出的包,记录物理层信息频偏和MAC地址;
		然后,我们将该帧的频偏值和MAC地址作为一个组合,前半部分数据用来训练机器学习模型,
    使用的机器学习模型有k最近邻算法(k-NearestNeighbor,KNN)、
    线性回归分析、随机森林算法、决策树回归、支持向量机(Support Vector Machine,SVM);
		最后,我们将记录到的后半部分数据作为测试集,用训练好的模型进行区分。
		训练和测试代码都由Python实现,使用的机器学习库是目前最常用的开源Python机器学习框架scikit-learn\cite{sklearn}。

    下面介绍测试结果。我们对三台设备进行监听,使用的训练集组数为1186组,多组数据求平均,
    频偏结果如\ref{tab:envaluate_freq_offset}所示。三部设备分别为华为荣耀6p、iPhone4s、小米Note。

    可以看到,不同设备之间的频偏差别很大,而且标准差不大,作为用来识别设备的参数非常合适。
    我们使用的测试集组数为594组,测试结果如表\ref{tab:envaluate_identify_sta}所示。
    可以看到,无论使用机器学习哪种模型,正确率都在96\%以上。

    以上实验表明,通过物理层信息频偏对不同设备进行识别,正确率较高,
    这种方法可以在多种场景下进行应用,例如:
			\begin{itemize}
				\item 区别不同的STA,使手机和笔记本电脑工作在STA模式,用GRTSEC对这些设备进行区分,
				应用场景是本节提到的防邻居“蹭网”;
				\item 区别不同的AP,在同一地点存在多个同SSID的AP,用GRTSEC进行区分,
				应用场景是识别\ref{subsec:demo_fake_wifi}节提出的伪装AP;
				\item 识别不同地点的AP,网络服务商在不同地点布置AP,例如校园网,用户设备可以通过AP进行定位。
			\end{itemize}

    上述样例使用频偏对不同设备进行识别,不区分AP和STA,实际上对于不同AP之间的识别,可以用时钟偏移这个物理层信息。
    有研究利用AP发出的beacon包的时间间隔,推算出不同AP设备的时钟偏移\cite{tmc10clock}。
    GRTSEC提供了精确到10ns的帧时间戳,完全可以支持此类研究。
		区别不同的STA时,由于STA不会定时发送beacon包,我们无法获得时钟偏移,只能得到频偏,无法使用此方法。

  \section{讨论与总结}\label{sec:envaluation_summary}
  本章对支持物理层WiFi安全研究的验证平台GRTSEC进行了评估,对测试结果进行了呈现。
  下面我们对GRTSEC满足设计目标进行论证。

  在第\ref{chap:demand}章我们针对三种WiFi安全研究,模拟WiFi攻击、基于加密的安全机制、基于身份验证的安全机制,
  总结了对无线验证平台的需求,提出了设计目标:可与商用802.11网卡实时通信,可自定义物理层和MAC层的数据处理流程,
  方便获取物理层信息,支持安全算法的软硬件实现,兼容上层网络协议。
  在前期工作GRT2.0系统中,通过高性能射频通信库的设计以及团队的工作,达到了与商用802.11网卡实时通信、可自定义物理层和MAC层的数据处理流程、
  兼容上层网络协议的目标。在GRTSEC的设计中,我们提供了一套物理层信息的提取与软硬件编程的框架,
  达到了方便获取物理层信息、支持安全算法的软硬件实现的目标。

  结合测试结果,\ref{sec:envaluation_performance}节的平台测试结果表明,GRTSEC提取物理层信息是高效且有效的,
  在占用资源较少、对软件性能影响低的前提下,提供了多种常见的物理层信息,在实际通信环境中收集这些物理层信息并进行了数据的简单处理。
  \ref{sec:envaluation_demo}节提供了两个使用样例,搭建伪装WiFi的使用样例进一步说明了GRTSEC可以模拟WiFi身份伪装攻击,
  兼容上层网络协议栈,利用物理层信息识别不同WiFi设备的样例对一个实际的WiFi安全研究进行验证,
  验证结果一方面说明这种安全策略是可行的,另一方面也论证了GRTSEC在安全研究中具有实际的使用价值。
  GRTSEC还提供了软硬件数据处理的框架,例如对于基于加密的安全机制,用户可以先通过软件验证加密算法正确性,再通过我们提供的硬件分析模块实现算法,
  对加密解密过程进行加速,在实际通信环境中加以验证。
  总的来说,GRTSEC作为一种支持物理层WiFi安全研究的验证平台,可以满足WiFi安全研究者的需求

	% 结论。
	% Copyright (c) 2008-2009 solvethis
% Copyright (c) 2010-2011 Casper Ti. Vector
% Public domain.

\chapter{结论与展望}
本章将在\ref{sec:conclusion}进行总结,在\ref{sec:future_work}进行未来工作展望。

  \section{总结}\label{sec:conclusion}
  本文首先对物理层WiFi安全的研究进行了深入的调研,总结了物理层WiFi安全研究对无线验证平台的需求,
  指出了现有无线验证平台的不足,提出了验证平台的设计目标。

  本文提出一种新型的支持物理层WiFi安全研究的验证平台GRTSEC,并且在北京大学可重构体系结构课题小组提出的GRT2.0系统的基础上加以实现。
  针对GRT2.0在物理层WiFi安全研究方面的几点不足,GRTSEC进行了相应的改进。

  为了更好的说明GRTSEC在物理层WiFi安全研究方面的优势,本文设计并实现了多个使用样例,
  实验表现使用样例在设备认证等方面改进了现有的WiFi安全机制,研究者也可以基于使用样例进行更深入的研究或开发。

  \section{未来工作展望}\label{sec:future_work}
  GRTSEC作为GRT2.0系统的扩展,一些特性将会在下一代GRT系统GRT3.0中加以应用,比如物理层信息从硬件逻辑到嵌入式软件的通道,
  因此未来的一项工作是将GRTSEC的特性合并到GRT主线的开发中。

  另外,我们发现开发硬件时调试困难,开发周期较长,研究者常常受困于安全算法的硬件实现,
  Xilinx开发套件提供的HLS(High-Level Synthesis,高层次综合)\cite{xilinxhls}工具可以有效地降低开发难度,提供开发效率。
  目前GRTSEC的框架是支持HLS的,将来希望针对HLS进行进一步地优化,比如提供适用于HLS的硬件模块接口。

  其次,在软件开发方面,我们在嵌入式软件中应用了Xilinx提供的xilkernal,这是一个简易的操作系统,提供pthread等多线程库,
  但是Xilinx在SDK 2017.1及之后的版本舍弃了xilkernal,转而支持Free RTOS,因此将来的一项工作是将GRTSEC的软件代码移植到新的嵌入式操作系统上。

  最后,根据GRT的新特性,GRTSEC也会提取更多的适用于安全研究的物理层信息,比如与多天线相关的信息。


	\begin{appendix}
		% 参考文献。
		\bibliography{ref/thesis}
		% 各附录。
		% Copyright (c) 2008-2009 solvethis
% Copyright (c) 2010-2011 Casper Ti. Vector
% Public domain.

\chapter{攻读硕士学位期间发表的论文及专利}
\raggedbottom % 避免某些奇怪的“Underfull \vbox”警告。
	\begin{description}
		\subsection*{已授权专利}
		\item{[1]} 王韬、\textbf{李晓光}、吴浩洋、吕松武,“一种基于FPGA的无线电光纤连接接口通信库及其实现方法”,
		授权日:2017年3月28日,专利号:201510239058.6
		\subsection*{已申请专利}
		\item{[2]} 王韬、\textbf{李晓光}、吴浩洋、吕松武,“一种利用物理层信息识别伪装WiFi的方法和系统”,
		申请日:2017年3月21日,申请号:201710169111.9
		\subsection*{已发表论文}
		\item{[3]} Tao Wang, Guangyu Sun, Jiahua Chen, Jian Gong, Haoyang Wu, \textbf{Xiaoguang Li}, Songwu Lu, and Jason Cong, "GRT: a Reconfigurable SDR Platform with High Performance and Usability," ACM SIGARCH Computer Architecture News (CAN), Vol. 42, No. 4, pp. 51-56, September 2014

		\item{[4]} Jiahua Chen, Tao Wang, Haoyang Wu, Jian Gong, \textbf{Xiaoguang Li}, Yang Hu, Gaohan Zhang, Zhiwei Li, Junrui Yang, and Songwu Lu, "A High-performance and High-programmability Reconfigurable Wireless Development Platform (Demonstration Paper)," in Proceedings of the 2014 International Conference on Field-Programmable Technology (ICFPT 2014), December 10-12, 2014, Shanghai, China, pp. 350-353.

		\item{[5]} Haoyang Wu, Tao Wang, Zhiwei Li, Boyan Ding, \textbf{Xiaoguang Li}, Tianfu Jiang, Jun Liu, and Songwu Lu, "GRT 2.0: An FPGA-based SDR Platform for Cognitive Radio Networks (Abstract Only)," in Proceedings of ACM/SIGDA International Symposium on Field-Programmable Gate Arrays (FPGA 2017), February 22-24, 2017, Monterey, CA, USA, pp. 294-295.

		\item{[6]} 吴浩洋、王韬、陈佳华、龚健、\textbf{李晓光}、张高瀚、吕松武, "GRT:高性能可定制无线网络底层软硬件开放平台," 电子科技大学学报, Vol. 44, No. 1, pp. 123-128, 2015年1月.

		\item{[7]} Yang Tian, Kaigui Bian, Guobin Shen, Xiaochen Liu, \textbf{Xiaoguang Li}, Thomas Moscibroda, "Contextual-code: Simplifying information pulling from targeted sources in physical world," in Computer Communications (INFOCOM), 2015 IEEE Conference on. IEEE, 2015: 2245-2253.

	\end{description}

\flushbottom % 取消 \raggedbottom 的作用。

	\end{appendix}

	% 以下为正文之后的部分。
	\backmatter

	% 致谢。
	% Copyright (c) 2008-2009 solvethis
% Copyright (c) 2010-2011 Casper Ti. Vector
% Public domain.

\chapter{致谢}
时光荏苒,岁月如梭,三年的硕士时光很快就要过去了,搭环境、写代码、调参数、读论文的场景还历历在目。
三年来我经历了挫折与历练,也经历了收获与喜悦,感谢导师、同学、朋友、家人的陪伴,在我困难时给予我支持,在我快乐时共同分享,
这些都将是我宝贵的财富,我会一直铭记。

首先,我要感谢我的导师王韬老师。王老师是我从大三到研究生期间最重要的引路人,不仅在学业、科研、工作上给予我启迪与帮助,还在做人做事上起到了表率作用。
王老师平易近人、身体力行,我还记得大四时王老师陪伴我们调试代码,教给我调试技巧,其兢兢业业的工作态度是激励我踏实勤奋的重要动力。
后来课题组日渐壮大,年轻学生不断加入,王老师告诉我们一个人能力再强,也无法独立完成大的工作,只有将学会的东西分享出去,互相学习、互相促进,
在一个团队中协力合作,凝聚力量,才能完成更大的工作。我学会了分享与学习,学会了相信同伴,这在我后来的科研和工作中受益匪浅。
王老师教育我们坚持手中的工作,勤奋努力,哪怕短时间内无法获得成效,长期的积累会带来坚实的收获,几位师兄师姐厚积薄发的科研和工作经历印证了这一点。
努力不一定得到回报,但不努力肯定没有回报,坚持努力总会获得回报。
王老师学识渊博,还不忘与时俱进、坚持读书与学习。王老师最初在体系结构方面颇有建树,这些年来又在软件工程、项目管理、人工智能等领域拓展自己的能力,
与学生一起重学概率论、研究小样本学习,给我留下很深的印象,学业的结束不代表学习的结束,我们只有坚持吸取养分,结合温故知新,才能跟上时代的步伐,与时俱进。
老师的悉心栽培恩重如山,非区区百字可以言表,我当铭记于心,砥砺前行。

然后,我要感谢吕松武老师。吕老师也指导了我五年的时间,虽然长期在美国,每年只回国数次,但每次都能为我们带来行业最新的进展,带来国际化的视野。
吕老师让我们看到了国际上最优秀的学生是如何学习和工作的,告诉我们怎样成为最优秀的学生。
除此之外,吕老师在生活和求职上也给予我巨大的帮助,与王老师一样,不仅是我学习和科研的导师,更是教给了我做人的道理,是我人生的导师。
我要感谢刘君老师,刘老师雷厉风行的做事风格为组里带来了执行力,并且在生活上给予我和我的妻子帮助,是我的良师益友。

此外,我要感谢我们实验室高能效计算与应用中心的同学们,我的工作离不开你们的帮助。
感谢龚健师兄,你的严谨认真的态度为组里的师弟师妹们带来表率,你的代码结构清晰、表意明确,论文严谨细致,是我们课题组的开拓者。
感谢陈佳华师姐,你工作勤奋、勇于探索、乐于创新,是组里的中坚力量,毕业后也不忘回报课题组。
感谢吴浩洋师兄,你是组里唯一的博士,是我的毕业论文最重要的协作者,你的沉稳、低调、务实帮助组里完成很多困难的工作,是真正的攻坚人。
感谢李志伟师弟和丁博岩师弟,在我做毕业设计的过程中给予我巨大的帮助和启发,是值得信任的伙伴。
感谢同级同学张高瀚、严磊,在一起做毕设的过程中是我站在同一战线的战友,讨论进展、相互支持。
感谢蒋天夫师弟、吴涵师妹、樊乃嘉师妹,你们为组里带来了活力和生机。
感谢同实验室的张宸师兄、王鹏师兄、魏学超师兄、王硕师兄、张炜其、姜双、王丰、肖倾城等,我在实验室得到了你们的不少帮助。

最后,我要感谢我的家人。谢谢妈妈在生活上的支持,多年来独自带我成长不易,养育之恩,寸心难报。
谢谢我的妻子周易,从相识到相爱到步入婚姻殿堂,一路上陪伴我共乘风雨、共享欢乐。

	% 原创性声明和使用授权说明。
	\include{chap/originauth}
\end{document}
