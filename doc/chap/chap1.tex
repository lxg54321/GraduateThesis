% Copyright (c) 2008-2009 solvethis
% Copyright (c) 2010-2011 Casper Ti. Vector
% Public domain.

\chapter{引言}
	\section{课题背景与研究意义}

		现代社会中,WiFi作为一种快速、便捷、不考虑流量的无线上网方式,成为人们基本的生活需求,WiFi接入点遍布家庭、企业、公共场所等区域。
		WiFi的普及为人们的网络生活带来了安全隐患,一些不法分子通过WiFi向用户发起攻击。
		相对于有线网络,WiFi网络的通信内容广播到空气中,附近任何人都可以获取并插入自己想发送的内容。
		现有的WiFi加密协议包括WEP、WPA和WPA2等存在安全漏洞,有被破解的可能\cite{wisec09wep}。
		有的攻击者破解个人WiFi的密码,目的是占用网络资源,降低合法用户的服务质量,俗称“蹭网”。
		另有一些攻击者利用带无线网卡的电脑和网络分析软件搭建钓鱼WiFi,监听网络报文,
		获取用户手机号、网站账号、密码等敏感信息,甚至可以窃取钱财,给用户带来巨大损失。
		在2015年的央视3·15晚会上,网络安全工程师伪装了演播室的免费WiFi,钓鱼得到的现场观众的信息,
		观众自拍的照片和邮箱密码竟出现在了演播室大屏幕上\cite{cctv315}。
		普通用户有时为了占用更多的无线网络带宽,也会有意或无意地向同网其它用户发起攻击,损害其它用户的网络质量\cite{tifs12reciprocity}。

		为了检测和阻止层出不穷的攻击,针对WiFi安全的研究广泛开展。
		近年来,结合物理层的安全机制成为WiFi安全研究的热点,物理层含有丰富的无线信道、设备位置、信号质量等信息,
		研究者探索如何利用这些信息加强WiFi的安全性,
		例如基于RSS(Received Signal Strength,接收信号强度)和CSI(Channel State Informatica,信道状态信息)的密钥生成策略\cite{access16key},
		基于RSS的物理层认证策略\cite{ieeewc10noncryp},
		基于CIR(Channel Impulse Response,信道冲激响应)的物理层认证技术\cite{milcom11cir, icc13cir},
		基于信道频率响应特性的物理层指纹和窃听检测技术\cite{icc07xiao, globecom10xiao}等。

		WiFi安全的研究需要有开发和验证平台,用以实现和评估研究成果。
		相对于上层的安全研究,物理层的安全研究对验证平台提出很高的要求,
		802.11a/g协议中规定的物理层数据传输速率为54Mbps,物理层延迟不超过16$\mu s$,软件实现的物理层无法在速率和延迟上满足要求,
		而硬件实现往往不可编程或可编程性较差,开发和调试周期长,缺少可供参考的使用样例等。
		从近些年的研究看,验证平台的质量常常成为制约研究成果有效性的瓶颈,
		一些验证平台不支持物理层的编程,一些验证平台延迟过高无法与商用WiFi设备实时通信\cite{mobicom13securearray}。
		“工欲善其事,必先利其器”,支持物理层编程的高性能WiFi验证平台成为WiFi安全研究必不可少的工具。

		现有支持物理层WiFi研究的无线开放平台按物理层实现方式和编程方式主要分为四类:
		\begin{itemize}
			\item 计算机仿真软件,没有射频前端,物理层由纯软件实现,支持软件编程,有较多的参考样例,代表是Matlab;
			\item 商用网卡,有射频前端,物理层由ASIC实现,不支持软件编程但支持软件配置,代表是Intel 5300网卡;
			\item 软件无线电平台,有射频前端,物理层由软件实现,支持软件编程,有较多的参考样例,代表是NI公司的硬件USRP与软件GNU Radio的组合;
			\item 基于FPGA的无线电开放平台,有射频前端,物理层由FPGA实现,部分支持编程,参考样例少,代表是莱斯大学的WARP平台。
		\end{itemize}

		各类平台在可编程性和性能上各有优势,但目前没有平台在性能和可编程性上同时很好地满足近年来物理层WiFi安全研究,
		具体体现在没有平台在满足802.11协议规定的速率和延迟的同时,可以提供给研究者物理层的高可编程性。
		因此,物理层WiFi安全的研究者需要一个无线开放平台,在性能和可编程性上同时满足研究需求,用以实现和验证研究内容。

	\section{主要研究内容}
		本文的基本目标是为物理层WiFi安全的研究者提供一个无线验证平台,
		围绕基本目标,本文的主要研究内容有以下几点:
		\begin{itemize}
			\item 对WiFi安全进行调研,总结常见的WiFi攻击以及WiFi安全相关的最新研究成果,分析WiFi安全研究对验证平台的需求;
			\item 对常见的WiFi验证平台进行调研,对各类平台进行对比和分析,找到各平台的优势以及不足;
			\item 设计适用于WiFi安全研究的无线验证平台,并基于FPGA进行实现;
			\item 为本文提出的验证平台设计WiFi安全相关的使用样例。
			% \item 使用本文提出的验证平台,搭建伪装WiFi,并提出一种新型的物理层WiFi识别技术识别不同WiFi设备。
		\end{itemize}

	\section{本文贡献}
		本文主要有以下贡献:
		\begin{itemize}
			\item 对物理层WiFi安全相关研究进行调研,分析了对WiFi验证平台的需求及现有系统的不足,
			定义了作为WiFi安全研究验证平台所需具有的特性;
			\item 在现有系统的基础上,设计并实现了一种支持物理层WiFi安全研究的验证平台GRTSEC(GRT for Security),
			性能上满足与商用设备通信的要求,提供易于编程调用的API;
			% 通过使用样例论证了本文提出的验证平台可以满足绝大多数物理层WiFi安全研究的需求;
			\item 作为使用样例,提出一种利用物理层信息对WiFi设备进行识别的技术,
			利用本文提出的验证平台,在与商用设备实时通信中验证了其有效性。
		\end{itemize}

	\section{本文组织}
	第\ref{chap:background}章对WiFi安全相关的背景进行介绍;
	第\ref{chap:demand}章对WiFi安全研究进行需求分析,提出了设计目标;
	第\ref{chap:grt2.0}章介绍本文的前期工作,GRT2.0系统,对本人负责的部分进行重点介绍;
	第\ref{chap:grtsec_design}章是支持物理层WiFi安全研究的验证平台GRTSEC的设计与实现;
	第\ref{chap:envaluation}章是性能测试与评估,结合使用样例对平台的性能进行详细测试,对结果进行分析;
	第\ref{chap:conclusion}章是文章总结与下一步研究的展望。
