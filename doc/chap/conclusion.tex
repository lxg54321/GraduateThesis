% Copyright (c) 2008-2009 solvethis
% Copyright (c) 2010-2011 Casper Ti. Vector
% Public domain.

\chapter{结论与展望}
本章将在\ref{sec:conclusion}进行总结,在\ref{sec:future_work}进行未来工作展望。

  \section{总结}\label{sec:conclusion}
  本文首先对物理层WiFi安全的研究进行了深入的调研,总结了物理层WiFi安全研究对无线验证平台的需求,
  指出了现有无线验证平台的不足,提出了验证平台的设计目标。

  本文提出一种新型的支持物理层WiFi安全研究的验证平台GRTSEC,并且在北京大学可重构体系结构课题小组提出的GRT2.0系统的基础上加以实现。
  针对GRT2.0在物理层WiFi安全研究方面的几点不足,GRTSEC进行了相应的改进。

  为了更好的说明GRTSEC在物理层WiFi安全研究方面的优势,本文设计并实现了多个使用样例,
  实验表现使用样例在设备认证等方面改进了现有的WiFi安全机制,研究者也可以基于使用样例进行更深入的研究或开发。

  \section{未来工作展望}\label{sec:future_work}
  GRTSEC作为GRT2.0系统的扩展,一些特性将会在下一代GRT系统GRT3.0中加以应用,比如物理层信息从硬件逻辑到嵌入式软件的通道,
  因此未来的一项工作是将GRTSEC的特性合并到GRT主线的开发中。

  另外,我们发现开发硬件时调试困难,开发周期较长,研究者常常受困于安全算法的硬件实现,
  Xilinx开发套件提供的HLS(High-Level Synthesis,高层次综合)\cite{xilinxhls}工具可以有效地降低开发难度,提供开发效率。
  目前GRTSEC的框架是支持HLS的,将来希望针对HLS进行进一步地优化,比如提供适用于HLS的硬件模块接口。

  其次,在软件开发方面,我们在嵌入式软件中应用了Xilinx提供的xilkernal,这是一个简易的操作系统,提供pthread等多线程库,
  但是Xilinx在SDK 2017.1及之后的版本舍弃了xilkernal,转而支持Free RTOS,因此将来的一项工作是将GRTSEC的软件代码移植到新的嵌入式操作系统上。

  最后,根据GRT的新特性,GRTSEC也会提取更多的适用于安全研究的物理层信息,比如与多天线相关的信息。
