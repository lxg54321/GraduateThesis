% Copyright (c) 2008-2009 solvethis
% Copyright (c) 2010-2011 Casper Ti. Vector
% Public domain.

\chapter{结论与展望}\label{chap:conclusion}
本章将在\ref{sec:conclusion}进行总结,在\ref{sec:future_work}进行未来工作展望。

  \section{总结}\label{sec:conclusion}
  本文首先对物理层WiFi安全的研究进行了深入的调研,总结了物理层WiFi安全研究对无线验证平台的需求,
  指出了现有无线验证平台的不足,提出了验证平台的设计目标。

  本文的前期工作GRT2.0系统是本课题组发布的一款高性能可重构的软件无线电通信平台,
  本文在第\ref{chap:grt2.0}章介绍了本人在GRT2.0中完成的工作,射频通信库等。
  GRT2.0系统可以满足本文设计目标中的与商用网卡实时通信以及与上层网络协议栈连接的要求,
	但在获取物理层信息、物理层编程性、软硬件模块替换等方面存在不足。
  本文在第\ref{chap:grtsec_design}章设计了一种新型的支持物理层WiFi安全研究的验证平台GRTSEC,
  针对GRT2.0在物理层WiFi安全研究方面的几点不足进行了相应的改进,
  提供了丰富且易用的物理层信息编程接口,对安全算法进行加速的硬件分析模块,改进了射频通信库。
  第\ref{chap:envaluation}章结合实际的测试数据和使用样例说明了GRTSEC满足物理层WiFi安全研究的需求,
  研究者也可基于使用样例进行更深入的研究或开发。

  \section{未来工作展望}\label{sec:future_work}
  GRTSEC作为GRT2.0系统的扩展,一些特性将会在下一代GRT系统中加以应用,比如物理层信息从硬件逻辑到嵌入式软件的通道,
  射频通信库的回环模式,因此未来的一项工作是将GRTSEC的特性合并到GRT主线的开发中。

  另外,我们发现开发硬件时调试困难,开发周期较长,研究者常常受困于安全算法的硬件实现,
  Xilinx开发套件提供的HLS(High-Level Synthesis,高层次综合)\cite{xilinxhls}工具可以有效地降低开发难度,提高开发效率。
  GRTSEC将来希望针对HLS进行进一步地优化,比如提供适用于HLS的硬件模块接口。

  其次,在软件开发方面,我们在嵌入式软件中应用了Xilinx提供的xilkernal,这是一个简易的操作系统,提供pthread等多线程库,
  但是Xilinx在SDK 2017.1及之后的版本舍弃了xilkernal,转而支持Free RTOS,因此将来的一项工作是将GRTSEC的软件代码移植到新的嵌入式操作系统上。

  最后,根据GRT的新特性,GRTSEC也会提取更多的适用于安全研究的物理层信息,比如与多天线相关的信息。
