% Copyright (c) 2008-2009 solvethis
% Copyright (c) 2010-2011 Casper Ti. Vector
% Public domain.

\chapter{致谢}
时光荏苒,岁月如梭,三年的硕士时光很快就要过去了,搭环境、写代码、调参数、读论文的场景还历历在目。
三年来我经历了挫折与历练,也经历了收获与喜悦,感谢导师、同学、朋友、家人的陪伴,在我困难时给予我支持,在我快乐时共同分享,
这些都将是我宝贵的财富,我会一直铭记。

首先,我要感谢我的导师王韬老师。王老师是我从大三到研究生期间最重要的引路人,不仅在学业、科研、工作上给予我启迪与帮助,还在做人做事上起到了表率作用。
王老师平易近人、身体力行,我还记得大四时王老师陪伴我们调试代码,教给我调试技巧,其兢兢业业的工作态度是激励我踏实勤奋的重要动力。
后来课题组日渐壮大,年轻学生不断加入,王老师告诉我们一个人能力再强,也无法独立完成大的工作,只有将学会的东西分享出去,互相学习、互相促进,
在一个团队中协力合作,凝聚力量,才能完成更大的工作。我学会了分享与学习,学会了相信同伴,这在我后来的科研和工作中受益匪浅。
王老师教育我们坚持手中的工作,勤奋努力,哪怕短时间内无法获得成效,长期的积累会带来坚实的收获,几位师兄师姐厚积薄发的科研和工作经历印证了这一点。
努力不一定得到回报,但不努力肯定没有回报,坚持努力总会获得回报。
王老师学识渊博,还不忘与时俱进、坚持读书与学习。王老师最初在体系结构方面颇有建树,这些年来又在软件工程、项目管理、人工智能等领域拓展自己的能力,
与学生一起重学概率论、研究小样本学习,给我留下很深的印象,学业的结束不代表学习的结束,我们只有坚持吸取养分,结合温故知新,才能跟上时代的步伐,与时俱进。
老师的悉心栽培恩重如山,非区区百字可以言表,我当铭记于心,砥砺前行。

然后,我要感谢吕松武老师。吕老师也指导了我五年的时间,虽然长期在美国,每年只回国数次,但每次都能为我们带来行业最新的进展,带来国际化的视野。
吕老师让我们看到了国际上最优秀的学生是如何学习和工作的,告诉我们怎样成为最优秀的学生。
除此之外,吕老师在生活和求职上也给予我巨大的帮助,与王老师一样,不仅是我学习和科研的导师,更是教给了我做人的道理,是我人生的导师。
我要感谢刘君老师,刘老师雷厉风行的做事风格为组里带来了执行力,并且在生活上给予我和我的妻子帮助,是我的良师益友。

此外,我要感谢我们实验室高能效计算与应用中心的同学们,我的工作离不开你们的帮助。
感谢龚健师兄,你的严谨认真的态度为组里的师弟师妹们带来表率,你的代码结构清晰、表意明确,论文严谨细致,是我们课题组的开拓者。
感谢陈佳华师姐,你工作勤奋、勇于探索、乐于创新,是组里的中坚力量,毕业后也不忘回报课题组。
感谢吴浩洋师兄,你是组里唯一的博士,是我的毕业论文最重要的协作者,你的沉稳、低调、务实帮助组里完成很多困难的工作,是真正的攻坚人。
感谢李志伟师弟和丁博岩师弟,在我做毕业设计的过程中给予我巨大的帮助和启发,是值得信任的伙伴。
感谢同级同学张高瀚、严磊,在一起做毕设的过程中是我站在同一战线的战友,讨论进展、相互支持。
感谢蒋天夫师弟、吴涵师妹、樊乃嘉师妹,你们为组里带来了活力和生机。
感谢同实验室的张宸师兄、王鹏师兄、魏学超师兄、王硕师兄、张炜其、姜双、王丰、肖倾城等,我在实验室得到了你们的不少帮助。

最后,我要感谢我的家人。谢谢妈妈在生活上的支持,多年来独自带我成长不易,养育之恩,寸心难报。
谢谢我的妻子周易,从相识到相爱到步入婚姻殿堂,一路上陪伴我共乘风雨、共享欢乐。
