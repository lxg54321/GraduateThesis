% Copyright (c) 2008-2009 solvethis
% Copyright (c) 2010-2011 Casper Ti. Vector
% Public domain.

\chapter{需求分析与设计目标}\label{chap:demand}
本章将分析物理层WiFi安全研究对验证平台的需求,提出本文的设计目标。
本章首先在\ref{sec:platform_demand}节分析物理层WiFi安全研究对验证平台的需求,
然后在\ref{sec:design_goal}节提出本文的设计目标。

	\section{安全研究对验证平台的需求}\label{sec:platform_demand}
	本节分析物理层WiFi安全研究对验证平台的需求。
	首先分析模拟WiFi攻击的需求,见\ref{subsec:demand_perform_attack}小节,
	然后研究安全机制的需求,分为基于加密和基于非加密两类,基于加密的安全机制见\ref{subsec:demand_encryption}小节,
	基于非加密的安全机制主要是基于身份验证,见\ref{subsec:demand_identification}小节。

		\subsection{模拟WiFi攻击}\label{subsec:demand_perform_attack}
		\ref{sec:security_research}节曾介绍过,WiFi攻击有多种类型,伪装、DoS攻击、密钥破解等。
		模拟不同的WiFi攻击对验证平台有一些共同的需求,可以总结为以下几点。
		\begin{itemize}
			\item 可监听报文,可兼容上层网络分析工具。
			监听报文是网络设备的基本功能,而上层网络分析工具在发起WiFi攻击时很常用,例如使用网络工具Aircrack-ng破解WEP、WPA、WPA2加密\cite{ijarcet12wpa2}。
			\item 接收端提供常见的物理层信息,如CSI、RSSI等。
			一些攻击者除了需要监听合法设备报文内容外,还需要获取接收信号的物理层信息,用来推测用户所发内容\cite{ccs16csi}或破解密钥\cite{cns15key}等。
			\item 发送端任意修改MAC层帧内容,包括IP地址、MAC地址、AP模式下的SSID等。
			自定义帧内容是伪装攻击的前提,如需要修改源MAC地址和目的MAC地址发起DoS攻击\cite{wisec06spoofing};
			\item 可与商用设备实时通信。攻击对象是商用设备,如果验证平台不能与商用设备实时通信,则可发起的攻击会很受限。
		\end{itemize}

		\subsection{基于加密的安全机制}\label{subsec:demand_encryption}
		对传输内容进行加密是网络安全中常采取的措施,攻击者虽然可以“听”到内容但却无法理解。
		基于加密的安全机制分成几个关键步骤,密钥生成、密钥分发、加密解密。
		在\ref{subsec:80211security}小节提到,现有的WiFi加密协议有漏洞,研究者利用物理层信息对加密协议进行加强,对验证平台提出以下需求。
		\begin{itemize}
			\item 接收端提供常见的物理层信息,如CSI、RSSI等。
			在基于加密的WiFi安全研究中或利用CSI生成密钥\cite{infocom13key}、\cite{iwqos14key},
			或利用RSSI生成密钥\cite{mobicom08key}、\cite{mobicom09key}、\cite{tmc13key},
			也有文章研究了CSI和RSSI生成密钥的性能\cite{access16key}。
			表\ref{tab:key_generation_summary}总结了现有密钥生成的研究所需的物理层信息。
			\item 可兼容上层网络分析工具。有研究需要使用wireshark进行抓包\cite{tmc13key}。
			\item 可在实时通信中验证,通信双方达成密钥一致性。有研究使用WARP平台进行验证\cite{access16key},
			但由于WARP平台延迟高,无法满足密钥一致性的要求。
			\item 支持软硬件加密解密模块替换。加密解密过程需要密集的计算,软件可以快速实现加密算法,硬件可以提高效率,满足协议要求的时序。
		\end{itemize}
		\begin{table}[!hbp]
		\centering
		\caption{现有基于物理层信息的密钥生成系统总结}
		\label{tab:key_generation_summary}
			\begin{tabular}{|l|l|l|l|} \hline
			现有工作 & 物理层信息 & 具体技术 & 使用的验证平台 \\ \hline
			INFOCOM 13\cite{infocom13key} & CSI & 加时间戳,信道补偿 & Intel 5300 NIC \\ \hline
			IWQoS 14\cite{iwqos14key} & CSI & 联合多子载波,哈希 & Intel 5300 NIC \\ \hline
			TMC 13\cite{tmc13key} & RSSI & 信息重构 & Atheros AR 5B95 \\ \hline
			Access 16\cite{access16key} & RSSI CSI & 多环境测试分析结果 & WARPv3 \\ \hline
			\end{tabular}
		\end{table}

		\subsection{基于身份验证的安全机制}\label{subsec:demand_identification}
		基于加密的安全机制有一定局限性,复杂的加密算法难以在实际通信系统中实现\cite{ccs16mumimo},
		因此一些研究者探索非加密的安全机制,即基于身份验证的安全机制\cite{ieeewc10noncryp}。
		对验证平台提出以下需求。
		\begin{itemize}
			\item 接收端提供常见的物理层信息,如CSI、RSSI等,除此之外,还支持扩展物理层信息。
			例如使用频偏、信号相关性和星座点偏移\cite{mobicom08radiometric},
			使用CIR\cite{milcom11cir}。
			表\ref{tab:phy_auth_summary}总结了现有物理层身份验证的研究所需的物理层信息。
			\item 自定义物理层数据处理流程。自定义物理层帧格式和物理层控制逻辑,
			例如有研究在物理层发送端引入了人为的频偏\cite{asiaccs14csi},只有支持物理层编程的平台可以做到这一点。
			\item 可与商用设备实时通信,增加验证结果的可信度。
			\item 支持软硬件物理层信息处理模块替换。对物理层信息的处理过程需要密集的计算,软件可以快速实现加密算法,硬件可以提高效率,满足协议要求的时序。
		\end{itemize}
		\begin{table}[!hbp]
		\centering
		\caption{现有基于物理层信息的身份验证系统总结}
		\label{tab:phy_auth_summary}
			\begin{tabular}{|p{0.2\textwidth}|l|p{0.2\textwidth}|p{0.3\textwidth}|} \hline
			现有工作 & 物理层信息 & 具体技术 & 使用的验证平台 \\ \hline
			IEEE TPDS 13\cite{tpds13spoofing} & RSS & 聚类分析,支持向量机 & Orinoco silver card,Atheros miniPCI NIC \\ \hline
			IEEE Wireless Commun 10\cite{ieeewc10noncryp} & RSS & RSS序列反馈 & Dell Latitude E5400 laptop \\ \hline
			TVT 16\cite{tvt16spoofing} & RSSI & 博弈论,增强学习 & USRP N210 \\ \hline
			MobiSys 10\cite{mobisys10proximity} & RSS & 临近认证 & Nokia N800 Internet Tablets \\ \hline
			MILCOM 11\cite{milcom11cir} & CIR & 噪声消除 & 仿真软件 \\ \hline
			ICC 13\cite{icc13cir} & CIR & 利用多径延迟 & 仿真软件 \\ \hline
			INFOCOM 13\cite{infocom13csi} & CSI & CSI精度增强 & Intel IWL5300 NIC \\ \hline
			ASIA CCS 14\cite{asiaccs14csi} & 频偏 & 引入人为频偏 & USRP \\ \hline
			ICC 07\cite{icc07xiao} & 频偏 & 多径效应 & 仿真软件WiSE\cite{bell95wise} \\ \hline
			TMC 10\cite{tmc10clock} & 时钟偏移 & 时间戳 & Linksys WPC 55AG, Intel 3945ABG \\ \hline
			\end{tabular}
		\end{table}

	\section{设计目标}\label{sec:design_goal}
	综合以上模拟WiFi攻击、基于加密的安全机制、基于身份验证的安全机制对验证平台的需求,
	为了实现一种支持物理层WiFi安全研究的验证平台,本文对验证平台提出了如下设计目标:
	\begin{itemize}
		\item 实现802.11协议,可与商用无线网卡实时通信;
		\item 支持自定义数据处理流程,包括物理层和MAC层;
		\item 方便获取常用物理层信息,如RSSI、CSI、频偏等,支持用户扩展物理层信息;
		\item 支持安全算法的软件和硬件实现,提供软硬件模块互换的能力;
		\item 可与上层网络协议栈连接,兼容常见网络分析工具。
	\end{itemize}

	实现上述目标一个挑战是如何保证性能的同时,提供物理层足够高的可编程性。
	一般而言,可编程性较高的平台性能较差,比如第二章提到的GNU Radio,使用C++和Python编程,
	提供方便的可视化编程界面,但吞吐率和延时无法达到802.11a/g的要求。
	性能满足要求的平台一般可编程性差,甚至不具备物理层的可编程能力,
	商用网卡的性能最高,可以完全满足协议要求,但不具备物理层的可编程性。
	本文基于北京大学的GRT系统\cite{can14grt}进行开发,提出支持物理层WiFi安全研究的验证平台GRTSEC(GRT for Security)。
	GRT系统同时也是本文的前期工作,在第\ref{chap:grt2.0}章会部分进行介绍。
