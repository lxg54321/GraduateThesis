% Copyright (c) 2008-2009 solvethis
% Copyright (c) 2010-2011 Casper Ti. Vector
% Public domain.

\chapter{攻读硕士学位期间发表的论文及专利}
\raggedbottom % 避免某些奇怪的“Underfull \vbox”警告。
	\begin{description}
		\subsection*{已授权专利}
		\item{[1]} 王韬、\textbf{李晓光}、吴浩洋、吕松武,“一种基于FPGA的无线电光纤连接接口通信库及其实现方法”,
		授权日:2017年3月28日,专利号:201510239058.6
		\subsection*{已申请专利}
		\item{[2]} 王韬、\textbf{李晓光}、吴浩洋、吕松武,“一种利用物理层信息识别伪装WiFi的方法和系统”,
		申请日:2017年3月21日,申请号:201710169111.9
		\subsection*{已发表论文}
		\item{[3]} Tao Wang, Guangyu Sun, Jiahua Chen, Jian Gong, Haoyang Wu, \textbf{Xiaoguang Li}, Songwu Lu, and Jason Cong, "GRT: a Reconfigurable SDR Platform with High Performance and Usability," ACM SIGARCH Computer Architecture News (CAN), Vol. 42, No. 4, pp. 51-56, September 2014

		\item{[4]} Jiahua Chen, Tao Wang, Haoyang Wu, Jian Gong, \textbf{Xiaoguang Li}, Yang Hu, Gaohan Zhang, Zhiwei Li, Junrui Yang, and Songwu Lu, "A High-performance and High-programmability Reconfigurable Wireless Development Platform (Demonstration Paper)," in Proceedings of the 2014 International Conference on Field-Programmable Technology (ICFPT 2014), December 10-12, 2014, Shanghai, China, pp. 350-353.

		\item{[5]} Haoyang Wu, Tao Wang, Zhiwei Li, Boyan Ding, \textbf{Xiaoguang Li}, Tianfu Jiang, Jun Liu, and Songwu Lu, "GRT 2.0: An FPGA-based SDR Platform for Cognitive Radio Networks (Abstract Only)," in Proceedings of ACM/SIGDA International Symposium on Field-Programmable Gate Arrays (FPGA 2017), February 22-24, 2017, Monterey, CA, USA, pp. 294-295.

		\item{[6]} 吴浩洋、王韬、陈佳华、龚健、\textbf{李晓光}、张高瀚、吕松武, "GRT:高性能可定制无线网络底层软硬件开放平台," 电子科技大学学报, Vol. 44, No. 1, pp. 123-128, 2015年1月.

		\item{[7]} Yang Tian, Kaigui Bian, Guobin Shen, Xiaochen Liu, \textbf{Xiaoguang Li}, Thomas Moscibroda, "Contextual-code: Simplifying information pulling from targeted sources in physical world," in Computer Communications (INFOCOM), 2015 IEEE Conference on. IEEE, 2015: 2245-2253.

	\end{description}

\flushbottom % 取消 \raggedbottom 的作用。
